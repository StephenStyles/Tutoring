\documentclass[12pt]{article}

\usepackage{bm,amsthm,amssymb,amsfonts,mathtools}
\usepackage[margin=1in]{geometry} 
\usepackage{dsfont}
% neural net drawing
\usepackage{tikz}
\usetikzlibrary{positioning}

\usepackage{neuralnetwork}

% macros
\newcommand{\norm}[1]{\left\lVert#1\right\rVert}

\usepackage{amsmath,amssymb,amsthm,amsfonts,amscd}
\usepackage{showlabels}
\usepackage{color}
\usepackage{hyperref}
\usepackage[numeric]{amsrefs}
\usepackage{graphicx}

\usepackage[framemethod=TikZ]{mdframed}
%\usepackage[framemethod=default]{mdframed}


%\mdfdefinestyle{box}{%
%rightline=true,
%innerleftmargin=10,
%innerrightmargin=10,
%frametitlerule=true,
%frametitlerulecolor=blue,
%frametitlebackgroundcolor=white,
%frametitlerulewidth=2pt}

\mdfdefinestyle{TheoremFrame}{%
    linecolor=blue,
    outerlinewidth=1,
    roundcorner=15,
    innertopmargin= \baselineskip,
    innerbottommargin= \baselineskip,
    innerrightmargin=10,
    innerleftmargin=10,
    backgroundcolor=white}
    
\mdfdefinestyle{ProofFrame}{%
    linecolor=red,
    outerlinewidth=1,
    roundcorner=15,
    innertopmargin= \baselineskip,
    innerbottommargin= \baselineskip,
    innerrightmargin=10,
    innerleftmargin=10,
    backgroundcolor=white}
    
    
\setlength{\oddsidemargin}{0cm}
\setlength{\evensidemargin}{0cm}
\setlength{\marginparwidth}{0in}
\setlength{\marginparsep}{0in}
\setlength{\marginparpush}{0in}
\setlength{\topmargin}{0in}
\setlength{\headheight}{0pt}
\setlength{\headsep}{0pt}
\setlength{\footskip}{.3in}
\setlength{\textheight}{9.2in}
\setlength{\textwidth}{6.0in}
\setlength{\parskip}{0.25pt}
\setlength{\parindent}{0.25in}

    
\newlength\tindent
\setlength{\tindent}{\parindent}
\setlength{\parindent}{0pt}
\renewcommand{\indent}{\hspace*{\tindent}}

\newtheorem*{nthm}{Theorem}
\newtheorem*{nlem}{Lemma}
\newtheorem*{nprop}{Proposition}
\newtheorem*{ncor}{Corollary}
\newtheorem*{nconj}{Conjecture}
\newtheorem*{nclaim}{Claim}
\theoremstyle{remark}
\newtheorem*{define}{Definition}
\newtheorem*{nrem}{Remarks}
\newtheorem*{notation}{Notation}
\newtheorem*{note}{Note}
\newtheorem*{ex}{Example}
\newtheorem*{imt}{Important}
\newtheorem*{fact}{Fact}
\newcommand{\vs}{\vspace{0.1in}}
\newcommand{\Lim}[1]{\raisebox{0.5ex}{\scalebox{0.8}{$\displaystyle \lim_{#1}\;$}}}
\title{Induction}
\author{Stephen Styles}
\begin{document}
\maketitle

\begin{enumerate}

\item Prove by induction that $\displaystyle{\sum_{k=1}^n k = \frac{n(n-1)}{2}}$
\vspace{6cm}
\item Prove by induction that $\displaystyle{\sum_{k=1}^n k^2 = \frac{n(n-1)(2n-1)}{6}}$

\newpage
\item Prove by induction that $n^3-n$ is divisible by $3$ for all positive integers.
\vspace{7cm}
\item Prove that the sequence $\displaystyle{x_{n+1} = \frac{x_n + \sqrt{3x_n}}{2}}$ is an increasing sequence where $x_1 =1 $.

\newpage
\item Prove that $n! > 2^n$ for all positive integers greater than or equal to $4$. 

\vspace{7cm}
\item Prove that for any real number $x>-1$ and any positive integer $n$,\\ $(1+x)^n \geq 1 + nx$.

\newpage
\item  Using induction, prove that the sequence  $\displaystyle{a_{n+1} = \frac{2a_n}{3+a_n}}$ is monotone with\\ $a_1 = 1$ and bounded below by $0$.
\vspace{7cm}
\item A sequence $\{ a_n \}$ is given by $a_1 = 2$, $\displaystyle{a_n = \sqrt{2+a_{n-1}}}$
\begin{enumerate}
\item Show by induction that $\{ a_n \}$ is increasing and bounded above by $3$.

\item Find $\displaystyle{ \lim_{a \rightarrow \infty} a_n}$.
\end{enumerate}
\end{enumerate}

\end{document}