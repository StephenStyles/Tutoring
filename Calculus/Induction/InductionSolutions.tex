\documentclass[12pt]{article}

\usepackage{bm,amsthm,amssymb,amsfonts,mathtools}
\usepackage[margin=1in]{geometry} 
\usepackage{dsfont}
% neural net drawing
\usepackage{tikz}
\usetikzlibrary{positioning}

\usepackage{neuralnetwork}

% macros
\newcommand{\norm}[1]{\left\lVert#1\right\rVert}

\usepackage{amsmath,amssymb,amsthm,amsfonts,amscd}
\usepackage{showlabels}
\usepackage{color}
\usepackage{hyperref}
\usepackage[numeric]{amsrefs}
\usepackage{graphicx}

\usepackage[framemethod=TikZ]{mdframed}
%\usepackage[framemethod=default]{mdframed}


%\mdfdefinestyle{box}{%
%rightline=true,
%innerleftmargin=10,
%innerrightmargin=10,
%frametitlerule=true,
%frametitlerulecolor=blue,
%frametitlebackgroundcolor=white,
%frametitlerulewidth=2pt}

\mdfdefinestyle{TheoremFrame}{%
    linecolor=blue,
    outerlinewidth=1,
    roundcorner=15,
    innertopmargin= \baselineskip,
    innerbottommargin= \baselineskip,
    innerrightmargin=10,
    innerleftmargin=10,
    backgroundcolor=white}
    
\mdfdefinestyle{ProofFrame}{%
    linecolor=red,
    outerlinewidth=1,
    roundcorner=15,
    innertopmargin= \baselineskip,
    innerbottommargin= \baselineskip,
    innerrightmargin=10,
    innerleftmargin=10,
    backgroundcolor=white}
    
    
\setlength{\oddsidemargin}{0cm}
\setlength{\evensidemargin}{0cm}
\setlength{\marginparwidth}{0in}
\setlength{\marginparsep}{0in}
\setlength{\marginparpush}{0in}
\setlength{\topmargin}{0in}
\setlength{\headheight}{0pt}
\setlength{\headsep}{0pt}
\setlength{\footskip}{.3in}
\setlength{\textheight}{9.2in}
\setlength{\textwidth}{6.0in}
\setlength{\parskip}{0.25pt}
\setlength{\parindent}{0.25in}

    
\newlength\tindent
\setlength{\tindent}{\parindent}
\setlength{\parindent}{0pt}
\renewcommand{\indent}{\hspace*{\tindent}}

\newtheorem*{nthm}{Theorem}
\newtheorem*{nlem}{Lemma}
\newtheorem*{nprop}{Proposition}
\newtheorem*{ncor}{Corollary}
\newtheorem*{nconj}{Conjecture}
\newtheorem*{nclaim}{Claim}
\theoremstyle{remark}
\newtheorem*{define}{Definition}
\newtheorem*{nrem}{Remarks}
\newtheorem*{notation}{Notation}
\newtheorem*{note}{Note}
\newtheorem*{ex}{Example}
\newtheorem*{imt}{Important}
\newtheorem*{fact}{Fact}
\newcommand{\vs}{\vspace{0.1in}}
\newcommand{\Lim}[1]{\raisebox{0.5ex}{\scalebox{0.8}{$\displaystyle \lim_{#1}\;$}}}
\title{Induction}
\author{Stephen Styles}
\begin{document}
\maketitle

\begin{enumerate}

\item Prove by induction that $\displaystyle{\sum_{k=1}^n k = \frac{n(n+1)}{2}}$
\begin{mdframed}[style=TheoremFrame]
\textit{Solution:}\\

Step 1: Base case where $k=1$
\begin{align*}
\sum_{k=1}^1 k = 1 &= \frac{1(1+1)}{2}\\
&= 1
\end{align*}
Step 2: Assume true for $n$\\

Step 3: Prove for $n+1$
\begin{align*}
\sum_{k=1}^{n+1} k &= \sum_{k=1}^{n} k + n+1\\
&= \frac{n(n+1)}{2} + n+1\\
&= \frac{n^2+n)}{2} + \frac{2n+2}{2}\\
&= \frac{n^2 + 3n + 2}{n}\\
&= \frac{(n+1)(n+2)}{2}\\
&= \frac{(n+1)\big((n+1)+1\big)}{2}
\end{align*}
Therefore by induction, $\displaystyle{\sum_{k=1}^n k = \frac{n(n+1)}{2}}$
\end{mdframed}
\newpage
\item Prove by induction that $\displaystyle{\sum_{k=1}^n k^2 = \frac{n(n+1)(2n+1)}{6}}$
\begin{mdframed}[style=TheoremFrame]
\textit{Solution:}\\

Step 1: Base case where $k=1$
\begin{align*}
\sum_{k=1}^1 k^2 = 1^2 &= \frac{1(1+1)(2\cdot1+1)}{6}\\
&= 1
\end{align*}
Step 2: Assume true for $n$\\

Step 3: Prove for $n+1$
\begin{align*}
\sum_{k=1}^{n+1} k^2 &= \sum_{k=1}^{n} k^2 + (n+1)^2\\
&= \frac{n(n+1)(2n+1)}{6} + (n+1)^2\\
&= (n+1)\bigg( \frac{n(2n+1)}{6} + \frac{6(n+1)}{6}\bigg)\\
&= (n+1)\bigg( \frac{2n^2+n)}{6} + \frac{6n+6}{6}\bigg)\\
&= (n+1)\bigg( \frac{2n^2+7n+6}{6}\bigg)\\
&= (n+1)\bigg( \frac{(n+2)(2n+3)}{6}\bigg)\\
&= \frac{(n+1)\big((n+1)+1\big)\big(2(n+1)+1\big)}{6}
\end{align*}
Therefore by induction, $\displaystyle{\sum_{k=1}^n k^2 = \frac{n(n+1)(2n+1)}{6}}$
\end{mdframed}
\newpage
\item Prove by induction that $n^3-n$ is divisible by $3$ for all positive integers.
\begin{mdframed}[style=TheoremFrame]
\textit{Solution:}\\

Step 1: Let $n=1$ then $1^3 - 1 = 0$ which is divisible by $3$.\\

Step 2: Assume true for $n$.\\

Step 3: Prove for $n+1$

\begin{align*}
(n+1)^3 - (n+1) &= (n^3 + 3n^2 + 3n + 1) - (n+1)\\
&= n^3 -n + 3n^2 + 3n
\end{align*}
By our assumption $n^3-n$ is divisible by $3$ and $ 3n^2 + 3n$ is divisible by $3$, therefore $n^3 -n + 3n^2 + 3n$ is divisible by $3$. Therefore $(n+1)^3 - (n+1)$ is divisible by $3$.\\

So by induction $n^3-n$ is divisible by $3$ for all positive integers.
\end{mdframed}
\newpage
\item Prove that the sequence $\displaystyle{x_{n+1} = \frac{x_n + \sqrt{3x_n}}{2}}$ is an increasing sequence where $x_1 =1 $.
\begin{mdframed}[style=TheoremFrame]
\textit{Solution:}\\

Step 1: Base case $x_2 =\displaystyle{ \frac{x_1 + \sqrt{3x_1}}{2} = \frac{1+\sqrt{3}}{2} > 1 = x_1}$\\

Step 2: Assume up $x_n > x_{n-1}$\\

Step 3: Prove $x_{n+1} > x_n$\\

Since: 
\begin{align*}
x_n &> x_{n-1}\\
\Rightarrow 3x_n &> 3x_{n-1}\\
\Rightarrow \sqrt{3x_n} &> \sqrt{3x_{n-1}}
\end{align*}
Because $\sqrt{x}$ is a monotonically increasing function.

Then
\begin{align*}
x_{n+1} &= \frac{x_n +  \sqrt{3x_n}}{2}\\
&> \frac{x_{n-1} +  \sqrt{3x_{n-1}}}{2} = x_n\\
\end{align*}
Therefore $x_{n+1} > x_n$. Thus by induction the sequence $\{ x_n \}$ is an increasing sequence.
\end{mdframed}
\newpage
\item Prove that $n! > 2^n$ for all positive integers greater than or equal to $4$. 

\begin{mdframed}[style=TheoremFrame]
\textit{Solution:}\\

Step 1: Base case $4! = 4 \cdot 3 \cdot 2 = 24$, and $2^4 = 16$. Therefore $4! > 2^4$.\\

Step 2: Assume up to $n$.\\

Step 3: Prove for $n+1$.\\

\begin{align*}
(n+1)! &= (n+1)\cdot n!\\
&> (n+1)\cdot 2^n\\
&> 2 \cdot 2^n\\
&> 2^{n+1}
\end{align*}
Therefore by induction $n! > 2^n$ for all positive integers greater than or equal to $4$. 
\end{mdframed}
\newpage
\item Prove that for any real number $x>-1$ and any positive integer $n$,\\ $(1+x)^n \geq 1 + nx$.
\begin{mdframed}[style=TheoremFrame]
\textit{Solution:}\\

Step 1: Base case for $n=1$, $(1+x)^1 = 1+x = 1 + 1\cdot x$.\\

Step 2: Assume up to $n$.\\

Step 3: Prove for $n+1$\\

\begin{align*}
(1+x)^{n+1} &= (1+x)^n(1+x)\\
&\geq (1+nx)(1+x)\\
&= 1 + x + nx + nx^2\\
&= 1 + (n+1)x + nx^2\\
&>  1 + (n+1)x \text{ since } nx^2>0
\end{align*}
Therefore by induction for any real number $x>-1$ and any positive\\ integer $n$, $(1+x)^n \geq 1 + nx$.
\end{mdframed}
\newpage
\item  Using induction, prove that the sequence  $\displaystyle{a_{n+1} = \frac{2a_n}{3+a_n}}$ is monotone with\\ $a_1 = 1$ and bounded below by $0$.
\begin{mdframed}[style=TheoremFrame]
\textit{Solution:}\\

Step 1: Base case
\begin{align*}
a_2 &= \frac{2a_1}{3+a_1}\\
&= \frac{2}{3+1}\\
&= \frac{1}{2}\\
&> 0 
\end{align*}

Step 2: Assume true for $n$, i.e. $a_n >0$.\\

Step 3: Prove $a_{n-1} > 0$
\begin{align*}
a_{n+1} = \frac{2a_n}{3+a_n}\\
\end{align*}
Where $2a_n > 0$ and $3+a_n>0$, therefore $a_{n+1}>0$. So by induction we know that all $a_n >0$.\\

To solve that the sequence is monotone we check
\begin{align*}
a_{n+1} - a_n &= \frac{2a_n}{3+a_n} - a_n\\
&= \frac{2a_n}{3+a_n}-\frac{3a_n + {a_n}^2}{3+a_n}\\
&= \frac{-a_n-{a_n}^2}{3+a_n}\\
&<0
\end{align*}
Therefore, we have a sequence $\{a_n\}$ where each $a_n >0$ and $a_{n-1} - a_n < 0$. So our sequence is monotonically decreasing.
\end{mdframed}
\newpage
\item A sequence $\{ a_n \}$ is given by $a_1 = \sqrt{2}$, $\displaystyle{a_n = \sqrt{2+a_{n-1}}}$
\begin{enumerate}
\item Show by induction that $\{ a_n \}$ is increasing and bounded above by $3$.

\begin{mdframed}[style=TheoremFrame]
\textit{Solution:}

Step 1: Base case $a_1 = \sqrt{2}$, $a_2 = \sqrt{2+ a_1} = \sqrt{2+\sqrt{2}}$\\
Thus $a_1 < a_2 < 3$.\\

Step 2: Assume $a_{n-1}<a_n<3$\\

Step 3: Prove $a_n < a_{n+1} < 3$
\begin{align*}
a_{n-1} &< a_{n} < 3\\
2+a_{n-1} &< 2+a_{n} < 2+3\\
\sqrt{2+a_{n-1}} &< \sqrt{2+a_{n}} < \sqrt{3+2}\\
a_{n}&<a_{n+1}<\sqrt{5}<3
\end{align*}
Therefore by induction we see that $\{ a_n \}$ is an increasing sequence bounded above by $3$.
\end{mdframed}
\item Find $\displaystyle{ \lim_{a \rightarrow \infty} a_n}$.
\begin{mdframed}[style=TheoremFrame]
\textit{Solution:}

Since our sequence is always increasing and has an upper bound, by the Monotone Convergence Theorem, a limit must exist.
\begin{align*}
\lim_{n\rightarrow \infty} a_n &= \lim_{n\rightarrow \infty} \sqrt{2+a_{n-1}}\\
&= \sqrt{2+\lim_{n\rightarrow \infty}a_{n-1}}\\
\Rightarrow L &= \sqrt{2+L}\\
\Rightarrow L^2 &= L+2\\
 \Rightarrow L^2 &- L -2 =0\\
\Rightarrow (L+1)&(L-2) =0
\end{align*}
Therefore, the limit either equals $-1$ or $2$. But since all sequence starts at $\sqrt{2}$ and is always increasing, we see that our limit must be $2$.

That is $\displaystyle{ \lim_{n\rightarrow \infty} a_n = 2}$.
\end{mdframed}
\end{enumerate}
\end{enumerate}

\end{document}