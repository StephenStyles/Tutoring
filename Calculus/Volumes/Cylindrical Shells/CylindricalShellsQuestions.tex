\documentclass[16pt]{article}

\usepackage{amsmath,amssymb,amsthm,amsfonts,amscd}
\usepackage{showlabels}
\usepackage{color}
\usepackage{hyperref}
\usepackage[numeric]{amsrefs}
\usepackage{graphicx}

\usepackage[framemethod=TikZ]{mdframed}
%\usepackage[framemethod=default]{mdframed}


%\mdfdefinestyle{box}{%
%rightline=true,
%innerleftmargin=10,
%innerrightmargin=10,
%frametitlerule=true,
%frametitlerulecolor=blue,
%frametitlebackgroundcolor=white,
%frametitlerulewidth=2pt}

\mdfdefinestyle{TheoremFrame}{%
    linecolor=blue,
    outerlinewidth=1,
    roundcorner=15,
    innertopmargin= \baselineskip,
    innerbottommargin= \baselineskip,
    innerrightmargin=10,
    innerleftmargin=10,
    backgroundcolor=white}
    
\mdfdefinestyle{ProofFrame}{%
    linecolor=red,
    outerlinewidth=1,
    roundcorner=15,
    innertopmargin= \baselineskip,
    innerbottommargin= \baselineskip,
    innerrightmargin=10,
    innerleftmargin=10,
    backgroundcolor=white}
    
    
\setlength{\oddsidemargin}{0cm}
\setlength{\evensidemargin}{0cm}
\setlength{\marginparwidth}{0in}
\setlength{\marginparsep}{0in}
\setlength{\marginparpush}{0in}
\setlength{\topmargin}{0in}
\setlength{\headheight}{0pt}
\setlength{\headsep}{0pt}
\setlength{\footskip}{.3in}
\setlength{\textheight}{9.2in}
\setlength{\textwidth}{6.0in}
\setlength{\parskip}{0.25pt}
\setlength{\parindent}{0.25in}

    
\newlength\tindent
\setlength{\tindent}{\parindent}
\setlength{\parindent}{0pt}
\renewcommand{\indent}{\hspace*{\tindent}}

\newtheorem*{nthm}{Theorem}
\newtheorem*{nlem}{Lemma}
\newtheorem*{nprop}{Proposition}
\newtheorem*{ncor}{Corollary}
\newtheorem*{nconj}{Conjecture}
\newtheorem*{nclaim}{Claim}
\theoremstyle{remark}
\newtheorem*{define}{Definition}
\newtheorem*{nrem}{Remarks}
\newtheorem*{notation}{Notation}
\newtheorem*{note}{Note}
\newtheorem*{ex}{Example}
\newtheorem*{imt}{Important}
\newtheorem*{fact}{Fact}
\newcommand{\vs}{\vspace{0.1in}}
\newcommand{\Lim}[1]{\raisebox{0.5ex}{\scalebox{0.8}{$\displaystyle \lim_{#1}\;$}}}


\title{Calculating Volumes Using Cylindrical Shells}

\author{Stephen Styles}


\begin{document}
\maketitle

The formula to calculate the volume using cylindrical shells is:
\begin{align*}
V&= \int_a^b 2\pi\, r(x) f(x) \, dx
\end{align*}
Where $r(x)$ is the radius to the line you are rotating about. Using we just consider rotations around the $y$-axis, so in this cause $r(x)=x$. However, if we consider rotations around the line $x=a$, then our radius would be $r(x)=x-a$.\\

Examples:
\begin{enumerate}
\item Find the volume of the solid generated by rotating the region bounded by the curves $y=x$ and $y=x^2$, about the $y$-axis.
\begin{mdframed}[style=TheoremFrame]
\textit{Solution:}
\begin{align*}
x&=x^2 \text{ if and only if } x=0,1
\end{align*}
\begin{align*}
V&= \int_0^1 2 \pi x (x-x^2) \, dx\\
&= 2 \pi \int_0^1 x^2 - x^3 \, dx\\
&= 2 \pi \bigg(\frac{x^3}{3} - \frac{x^4}{4}\bigg) \bigg|_0^1\\
&= 2 \pi \bigg(\frac{1}{3} - \frac{1}{4} \bigg)\\
&= \frac{\pi}{6}
\end{align*}
\end{mdframed}
\item Find the volume of the solid generated by rotating the region bounded by the curves $y=\sqrt{x}$ and $y=x$, about the line $x=-4$.
\begin{mdframed}[style=TheoremFrame]
\textit{Solution:}\\

\end{mdframed}
\item Find the volume of the solid generated by rotating the region bounded by the curves\\ $y=\frac{1}{x\sqrt{x^2+1}}$, and $x=1$, and $x=2$ about the line $y$-axis.
\begin{mdframed}[style=TheoremFrame]
\textit{Solution:}
\begin{align*}
V&= \int_1^2 2 \pi x \frac{1}{x\sqrt{x^2+1}} \, dx\\
&= 2\pi \int_0^1 \frac{1}{x^2+1} \, dx \\
&= 2\pi \text{sinh}^{-1}(x) \bigg|_1^2\\
&= 2\pi \big(\text{sinh}^{-1}(2)-\text{sinh}^{-1}(1)\big)
\end{align*}
\end{mdframed}
\end{enumerate}

\end{document}