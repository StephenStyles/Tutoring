\documentclass[16pt]{article}

\usepackage{amsmath,amssymb,amsthm,amsfonts,amscd}
\usepackage{showlabels}
\usepackage{color}
\usepackage{hyperref}
\usepackage[numeric]{amsrefs}
\usepackage{graphicx}

\usepackage[framemethod=TikZ]{mdframed}
%\usepackage[framemethod=default]{mdframed}


%\mdfdefinestyle{box}{%
%rightline=true,
%innerleftmargin=10,
%innerrightmargin=10,
%frametitlerule=true,
%frametitlerulecolor=blue,
%frametitlebackgroundcolor=white,
%frametitlerulewidth=2pt}

\mdfdefinestyle{TheoremFrame}{%
    linecolor=blue,
    outerlinewidth=1,
    roundcorner=15,
    innertopmargin= \baselineskip,
    innerbottommargin= \baselineskip,
    innerrightmargin=10,
    innerleftmargin=10,
    backgroundcolor=white}
    
\mdfdefinestyle{ProofFrame}{%
    linecolor=red,
    outerlinewidth=1,
    roundcorner=15,
    innertopmargin= \baselineskip,
    innerbottommargin= \baselineskip,
    innerrightmargin=10,
    innerleftmargin=10,
    backgroundcolor=white}
    
    
\setlength{\oddsidemargin}{0cm}
\setlength{\evensidemargin}{0cm}
\setlength{\marginparwidth}{0in}
\setlength{\marginparsep}{0in}
\setlength{\marginparpush}{0in}
\setlength{\topmargin}{0in}
\setlength{\headheight}{0pt}
\setlength{\headsep}{0pt}
\setlength{\footskip}{.3in}
\setlength{\textheight}{9.2in}
\setlength{\textwidth}{6.0in}
\setlength{\parskip}{0.25pt}
\setlength{\parindent}{0.25in}

    
\newlength\tindent
\setlength{\tindent}{\parindent}
\setlength{\parindent}{0pt}
\renewcommand{\indent}{\hspace*{\tindent}}

\newtheorem*{nthm}{Theorem}
\newtheorem*{nlem}{Lemma}
\newtheorem*{nprop}{Proposition}
\newtheorem*{ncor}{Corollary}
\newtheorem*{nconj}{Conjecture}
\newtheorem*{nclaim}{Claim}
\theoremstyle{remark}
\newtheorem*{define}{Definition}
\newtheorem*{nrem}{Remarks}
\newtheorem*{notation}{Notation}
\newtheorem*{note}{Note}
\newtheorem*{ex}{Example}
\newtheorem*{imt}{Important}
\newtheorem*{fact}{Fact}
\newcommand{\vs}{\vspace{0.1in}}
\newcommand{\Lim}[1]{\raisebox{0.5ex}{\scalebox{0.8}{$\displaystyle \lim_{#1}\;$}}}


\title{Calculating Volumes Using the Disc Method - Solutions}

\author{Stephen Styles}


\begin{document}
\maketitle

A line can be rotated around either the $x$-axis or the $y$-axis in order to create a solid object. From here we can calculate the volume of this solid. Recall that the area of a circle is $\pi r^2$. In this case, we will be calculating a bunch of areas with radius $r = f(x)$, then summing them up to calculate the volume. \\

The formula to calculate the volume of this solid is 
\begin{align*}
V &= \int_a^b \pi {[ f(x) ]}^2 \, dx.
\end{align*}

Examples:
\begin{enumerate}
\item Find the volume of the solid generated by rotating the line $f(x) = 3x^2 + 2$ around the $x$-axis, bounded by the $y$-axis and the line $x=1$.
\begin{mdframed}[style=TheoremFrame]
\textit{Solution:}
\begin{align*}
V&= \int_0^1 \pi \big(3x^2+2 \big)^2 \, dx\\
&= \pi \int_0^1 9x^4 + 12x^2 + 4 \, dx\\
&= \pi \big( \frac{9x^5}{5} + 4x^3 + 4x \big) \bigg|_0^1\\
&= \pi \big(\frac{9}{5} + 4 + 4 \big)\\
&= \frac{49\pi}{5}
\end{align*}
\end{mdframed}
\item Find the volume of the solid generated by rotating the line $f(x) = \frac{1}{x}$ around the $x$-axis bounded by the lines $x=1$ and $x=10$.
\begin{mdframed}[style=TheoremFrame]
\textit{Solution:}
\begin{align*}
V &= \int_1^{10} \pi \bigg(\frac{1}{x}\bigg)^2 \, dx\\
&= \pi \int_1^{10} \frac{1}{x^2} \, dx\\
&= \pi \bigg(\frac{-1}{x}\bigg) \bigg|_1^{10}\\
&= \pi \bigg( \frac{-1}{10}+1\bigg)\\
&= \frac{9\pi}{10}
\end{align*}
\end{mdframed}
\end{enumerate}

The disc method can also be extended to cases when we want to find the volume generated by the area between two curves. In that case we will calculate the volume generated by the top curve, then subtract the volume generated by the bottom curve. The formula for this solid volume is

\begin{align*}
V &= \pi \int_a^b  {[ f(x) ]}^2  - {[ g(x) ]}^2 \, dx.
\end{align*}

Examples:
\begin{enumerate}
\item Find the volume of the solid generated by rotating the area between the curves $y=2-x^2$ and $y=x^2$ around the $x$-axis. (Note: these curves intersect at $x=-1$ and $x=1$.)
\begin{mdframed}[style=TheoremFrame]
\textit{Solution:}
\begin{align*}
V &= \pi \int_{-1}^1 (2-x^2)^2 - (x^2)^2 \, dx\\
&= \pi \int_{-1}^1 4-2x^2+x^4 - x^4 \, dx\\
&= \pi \int_{-1}^1 4-2x^2 \, dx\\
&= \pi \big(4x-\frac{2x^3}{3}\big) \bigg|_{-1}^1\\
&= \pi \big(4-\frac{2}{3} - (-4+\frac{2}{3})\big)\\
&= (8-\frac{4}{3})\pi\\
&= \frac{20\pi}{3}
\end{align*}
\end{mdframed}

\item Find the volume of the solid generated by rotating the area between the curves $y=x^2$ and $y=\sqrt{x}$ around the $x$-axis, bounded by the $y$-axis and the intersection of the two curves. 
\begin{mdframed}[style=TheoremFrame]
\textit{Solution:}\\

$x^2 = \sqrt{x} \text{ if and only if } x=1,0.$\\
For $x \in {[0,1]}$ $\sqrt{x} \geq x^2$.
\begin{align*}
V&= \pi \int_0^1 (\sqrt{x})^2 - (x^2) \, dx\\
&= \pi \int_0^1 x - x^4 \, dx\\
&= \pi \bigg(\frac{x^2}{2} - \frac{x^5}{5} \bigg) \bigg|_0^1\\
&= \pi \bigg(\frac{1}{2} - \frac{1}{5}\bigg)\\
&= \frac{3\pi}{10}
\end{align*}
\end{mdframed}
\end{enumerate}
Questions:
\begin{enumerate}
\item Find the volume of the solid generated by rotating the area bounded by the curves $y=6-x^2$ and $y=2$ around the $x$-axis. (Note: the curves intersect the the points $(-2,2)$ and $(2,2)$.

\begin{mdframed}[style=TheoremFrame]
\textit{Solution}:\\

\begin{align*}
V&= \pi \int_{-2}^2 (6-x^2)^2 - 2^2 \, dx\\
&= \pi \int_{-2}^2 36 - 12x^2 + x^4 - 4 \, dx\\
&= \pi \int_{-2}^2 32 - 12x^2 + x^4  \, dx\\
&= \pi \bigg(32x - 4x^3 + \frac{x^5}{5} \bigg) \bigg|_{-2}^2\\
&= \frac{384\pi}{5}
\end{align*}
\end{mdframed}
\newpage
\item Find the volume of the solid generated by rotating the curve $f(x) = (x-1)^2+1$ around the $x$-axis, bounded by the curves $x=0$ and $x=1$.

\begin{mdframed}[style=TheoremFrame]
\textit{Solution}:\\

\begin{align*}
V &= \pi \int_0^1 \bigg( (x-1)^2 + 1 \bigg)^2 \, dx\\
&= \pi \int_0^1 (x-1)^4 + 2(x-1)^2 + 1 \, dx\\
&= \pi \int_0^1 x^4-4x^3+6x^2-4x+1 +2x^2-4x+2 +1 \, dx\\
&= \pi \int_0^1 x^4-4x^3+8x^2-8x+4 \, dx\\
&=\pi \bigg( \frac{x^5}{5} - x^4 + \frac{8x^3}{3}-4x^2+4 \bigg) \bigg|_0^1\\
&= \frac{28\pi}{15}
\end{align*}
\end{mdframed}
\newpage
\item Calculate the volume of the solid generated by the area bounded by the curves $y=2\sqrt{x}$, $y=6$, and the $y$-axis, about the horizontal line $y=7$.

\begin{mdframed}[style=TheoremFrame]
\textit{Solution}:\\

We need to figure out when the two curves intersect to give us the upper bound on our integral.
\begin{align*}
2\sqrt{x} &= 6\\
\Rightarrow \sqrt{x} &= 3\\
\Rightarrow x &= 9
\end{align*}
Therefore we get $f(x) = 7 - 2\sqrt{x}$ and $g(x) = 7-6 = 1$.
\begin{align*}
V&= \pi \int_0^9 (7-2\sqrt{x})^2 - 1^2 \, dx\\
&= \pi \int_0^9 48 - 4 \sqrt{x}+4x \, dx\\
&= \pi \bigg(48x - \frac{8x^{3/2}}{3}+2x^2 \bigg) \bigg|_0^9\\
&= 378\pi
\end{align*}
\end{mdframed}
\newpage\item Calculate the volume of the solid generated by rotating the region bounded by the curves\\ $\displaystyle{y=\bigg(\frac{x}{2}\bigg)^2}$, $y=4$, and $y=0$ about the line $x$-axis.
\begin{mdframed}[style=TheoremFrame]
\textit{Solution:}\\

We are given the horizontal $y$ bounds, but we need the vertical $x$ bounds.
\begin{align*}
 \bigg(\frac{x}{2}\bigg)^2 &= 4\\
\Rightarrow \frac{x}{2} &= 2\\
\Rightarrow x &= 4
\end{align*}
\begin{align*}
\bigg(\frac{x}{2}\bigg)^2 &= 0\\
\Rightarrow x &= 0
\end{align*}
\begin{align*}
V&= \pi \int_0^4  4^2 - \bigg(\bigg(\frac{x}{2}\bigg)^2 \bigg)^2 \, dx\\
&= \pi \int_0^4 16 -\frac{x^4}{16} \, dx\\
&= \bigg(16\pi x - \frac{\pi x^5}{80} \bigg) \bigg|_0^4\\
&= \frac{256\pi}{5}
\end{align*}
\end{mdframed}
\end{enumerate}
\end{document}