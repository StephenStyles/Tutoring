\documentclass[12pt]{article}

\usepackage{bm,amsthm,amssymb,amsfonts,mathtools}
\usepackage[margin=1in]{geometry} 
\usepackage{dsfont}
% neural net drawing
\usepackage{tikz}
\usetikzlibrary{positioning}

\usepackage{neuralnetwork}

% macros
\newcommand{\norm}[1]{\left\lVert#1\right\rVert}

\usepackage{amsmath,amssymb,amsthm,amsfonts,amscd}
\usepackage{showlabels}
\usepackage{color}
\usepackage{hyperref}
\usepackage[numeric]{amsrefs}
\usepackage{graphicx}

\usepackage[framemethod=TikZ]{mdframed}
%\usepackage[framemethod=default]{mdframed}


%\mdfdefinestyle{box}{%
%rightline=true,
%innerleftmargin=10,
%innerrightmargin=10,
%frametitlerule=true,
%frametitlerulecolor=blue,
%frametitlebackgroundcolor=white,
%frametitlerulewidth=2pt}

\mdfdefinestyle{TheoremFrame}{%
    linecolor=blue,
    outerlinewidth=1,
    roundcorner=15,
    innertopmargin= \baselineskip,
    innerbottommargin= \baselineskip,
    innerrightmargin=10,
    innerleftmargin=10,
    backgroundcolor=white}
    
\mdfdefinestyle{ProofFrame}{%
    linecolor=red,
    outerlinewidth=1,
    roundcorner=15,
    innertopmargin= \baselineskip,
    innerbottommargin= \baselineskip,
    innerrightmargin=10,
    innerleftmargin=10,
    backgroundcolor=white}
    
    
\setlength{\oddsidemargin}{0cm}
\setlength{\evensidemargin}{0cm}
\setlength{\marginparwidth}{0in}
\setlength{\marginparsep}{0in}
\setlength{\marginparpush}{0in}
\setlength{\topmargin}{0in}
\setlength{\headheight}{0pt}
\setlength{\headsep}{0pt}
\setlength{\footskip}{.3in}
\setlength{\textheight}{9.2in}
\setlength{\textwidth}{6.0in}
\setlength{\parskip}{0.25pt}
\setlength{\parindent}{0.25in}

    
\newlength\tindent
\setlength{\tindent}{\parindent}
\setlength{\parindent}{0pt}
\renewcommand{\indent}{\hspace*{\tindent}}

\newtheorem*{nthm}{Theorem}
\newtheorem*{nlem}{Lemma}
\newtheorem*{nprop}{Proposition}
\newtheorem*{ncor}{Corollary}
\newtheorem*{nconj}{Conjecture}
\newtheorem*{nclaim}{Claim}
\theoremstyle{remark}
\newtheorem*{define}{Definition}
\newtheorem*{nrem}{Remarks}
\newtheorem*{notation}{Notation}
\newtheorem*{note}{Note}
\newtheorem*{ex}{Example}
\newtheorem*{imt}{Important}
\newtheorem*{fact}{Fact}
\newcommand{\vs}{\vspace{0.1in}}
\newcommand{\Lim}[1]{\raisebox{0.5ex}{\scalebox{0.8}{$\displaystyle \lim_{#1}\;$}}}
\title{Sample Midterm}
\author{Stephen Styles}
\begin{document}
\maketitle

\begin{enumerate}

\item Find the limit if it exists:
\begin{enumerate}

\item $\displaystyle{\lim_{x \rightarrow -\infty} \frac{5x-3}{\sqrt{9x^2+6}}}$
\vspace{5cm}

\item $\displaystyle{\lim_{x \rightarrow 5} \frac{\sqrt{x^2+11}-6}{x-5}}$
\vspace{5cm}

\item $\displaystyle{\lim_{x \rightarrow 0} 6x^2 \cot(3x) \csc(2x)}$
\end{enumerate}
\newpage

\item Is the function $\displaystyle{ f(x) = \frac{1}{e^{2x}+\sin(x)-100}}$ continuous?
\vspace{5cm}

\item Find the domain of the function $\displaystyle{ f(x) = \frac{\ln(x-3)}{\sqrt{9-x}}}$
\vspace{5cm}

\item Show that $\displaystyle{\lim_{x \rightarrow 0^+} \sqrt{x}\bigg( 1 - \sin^2\bigg(\frac{2\pi}{x}\bigg)\bigg)=0}$
\newpage

\item \begin{enumerate}
\item State the formal definition of a derivative.
\vspace{3cm}
\item Using the formal definition of a derivative, find the derivative of $\displaystyle{f(x) = \sqrt{x-5}}$
\end{enumerate}
\newpage

\item Suppose c is a constant and the function f is given by:
\[ f(x) = \begin{cases} 
      c^x & x<1 \\
      3cx-2 & x \geq 1 \\
   \end{cases}
\]
\begin{enumerate}
\item Calculate $\displaystyle{ \lim_{x \rightarrow 1^-} f(x)}$ and $\displaystyle{ \lim_{x \rightarrow 1^+} f(x)}$.
\item Find all values of $c$ so that the function $f(x)$ is continuous everywhere.
\end{enumerate}
\newpage
\item Find the derivative of the following functions but \textbf{DO NOT} simplify your answers:
\begin{enumerate}
\item $\displaystyle{ f(x) = \frac{e^x}{\sin(x^2)}}$
\vspace{6cm}

\item $\displaystyle{ f(x) = 2^{\csc(x)} \cdot \cot(-5x)}$
\vspace{6cm}

\item $\displaystyle{ f(x) = \sec^2\bigg(e^{\tan(\sqrt{x})}\bigg)}$
\end{enumerate}
\newpage
\item A poster is to have an area of $600 \text{ in}^2$ with $1$ inch margins at
the bottom and sides and a $2$ inch margin on top. What dimensions will
give the largest printed area?
\newpage
\item Use implicit differentiation to find the derivative of $ \displaystyle{ x\cdot \tan(y) = xy+\cos(xy)}$
\vspace{8cm}
\item Find the equation of the tangent line to the graph of $y = \tan(2x) + 3\sec(x)$ at the point $(0,3)$.
\newpage

\item Make a detailed graph of $\displaystyle{ f(x) = \frac{4x}{x^2+1}}$
\end{enumerate}


\end{document}