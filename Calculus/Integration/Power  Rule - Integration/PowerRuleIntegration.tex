\documentclass[12pt]{article}

\usepackage{bm,amsthm,amssymb,amsfonts,mathtools}
\usepackage[margin=1in]{geometry} 
\usepackage{dsfont}
% neural net drawing
\usepackage{tikz}
\usetikzlibrary{positioning}

\usepackage{neuralnetwork}

% macros
\newcommand{\norm}[1]{\left\lVert#1\right\rVert}

\usepackage{amsmath,amssymb,amsthm,amsfonts,amscd}
\usepackage{showlabels}
\usepackage{color}
\usepackage{hyperref}
\usepackage[numeric]{amsrefs}
\usepackage{graphicx}

\usepackage[framemethod=TikZ]{mdframed}
%\usepackage[framemethod=default]{mdframed}


%\mdfdefinestyle{box}{%
%rightline=true,
%innerleftmargin=10,
%innerrightmargin=10,
%frametitlerule=true,
%frametitlerulecolor=blue,
%frametitlebackgroundcolor=white,
%frametitlerulewidth=2pt}

\mdfdefinestyle{TheoremFrame}{%
    linecolor=blue,
    outerlinewidth=1,
    roundcorner=15,
    innertopmargin= \baselineskip,
    innerbottommargin= \baselineskip,
    innerrightmargin=10,
    innerleftmargin=10,
    backgroundcolor=white}
    
\mdfdefinestyle{ProofFrame}{%
    linecolor=red,
    outerlinewidth=1,
    roundcorner=15,
    innertopmargin= \baselineskip,
    innerbottommargin= \baselineskip,
    innerrightmargin=10,
    innerleftmargin=10,
    backgroundcolor=white}
    
    
\setlength{\oddsidemargin}{0cm}
\setlength{\evensidemargin}{0cm}
\setlength{\marginparwidth}{0in}
\setlength{\marginparsep}{0in}
\setlength{\marginparpush}{0in}
\setlength{\topmargin}{0in}
\setlength{\headheight}{0pt}
\setlength{\headsep}{0pt}
\setlength{\footskip}{.3in}
\setlength{\textheight}{9.2in}
\setlength{\textwidth}{6.0in}
\setlength{\parskip}{0.25pt}
\setlength{\parindent}{0.25in}

    
\newlength\tindent
\setlength{\tindent}{\parindent}
\setlength{\parindent}{0pt}
\renewcommand{\indent}{\hspace*{\tindent}}

\newtheorem*{nthm}{Theorem}
\newtheorem*{nlem}{Lemma}
\newtheorem*{nprop}{Proposition}
\newtheorem*{ncor}{Corollary}
\newtheorem*{nconj}{Conjecture}
\newtheorem*{nclaim}{Claim}
\theoremstyle{remark}
\newtheorem*{define}{Definition}
\newtheorem*{nrem}{Remarks}
\newtheorem*{notation}{Notation}
\newtheorem*{note}{Note}
\newtheorem*{ex}{Example}
\newtheorem*{imt}{Important}
\newtheorem*{fact}{Fact}
\newcommand{\vs}{\vspace{0.1in}}
\newcommand{\Lim}[1]{\raisebox{0.5ex}{\scalebox{0.8}{$\displaystyle \lim_{#1}\;$}}}
\title{Power Rule - Integration}
\author{Stephen Styles}
\begin{document}
\maketitle

\begin{abstract}
If there are any mistakes in this worksheet/solution please email me at sjstyles@ualberta.ca. As well, if this worksheet does not cover enough details for the specific topic, notify me and I can expand it to include more questions.
\end{abstract}

Let us define
\begin{align*}
F(x) &= \int f(x) dx
\end{align*}
then,
\begin{align*}
\frac{d}{dx} F(x) = f(x).
\end{align*}
Some properties of integration:
\begin{enumerate}
\item $\displaystyle{ \int c f(x)dx = c \int f(x) dx}$ for all constants $c$.

\item $\displaystyle{ \int f(x) + g(x) dx = \int f(x) dx + \int g(x) dx}$.

\item $\displaystyle{ \int c \text{ } dx = cx}$
\end{enumerate}
The power rule for integration works in the opposite direction as it does for differentiation. If you recall:
\begin{align*}
\frac{d}{d x} x^n = n x^{n-1}
\end{align*}
Then the power rule for integration is defined as following:
\begin{align*}
\int x^n dx = \frac{1}{n+1} x^{n+1} + c \text{,    for all } n \not = -1
\end{align*}
If $n= -1$ then the integral becomes:
\begin{align*}
\int x^{-1} dx = \int \frac{1}{x} dx = \ln( |x| ) + c
\end{align*}
\newpage
Remember, we must include the "+c" at the end because unless we have more information about the function (i.e. initial values), if we differentiate $F(x)$, the constant $c$ will just disappear and we will regain $f(x)$ for any value of $c$.\\

In our previous definition for the power rule, we see that
\begin{align*}
\frac{d}{dx} \bigg(\frac{1}{n+1} x^{n+1} + c \bigg) &= \frac{1}{n+1}\bigg(\frac{d}{dx} x^{n+1} \bigg) + \frac{d}{dx}c\\
&= \frac{1}{n+1} \big( (n+1) x^n\big)\\
&= x^n
\end{align*}
Which is what we want since $\displaystyle{ \frac{d}{dx} F(x) = f(x)}$.\\

\begin{mdframed}[style=TheoremFrame]
\textit{Example:}\\

Find $\displaystyle{ \int 3x^5 - 2x^\frac{\pi}{2} + x -7 + \frac{2}{x} - \frac{6}{x^3}dx}$\\

\textit{Solution:}\\
\begin{align*}
\int 3x^5 - 2x^{\frac{\pi}{2}} + x -7 + \frac{2}{x} - \frac{6}{x^3} dx &= 3\bigg( \frac{1}{5+1} x^{5+1} \bigg) -2 \bigg(\frac{1}{\frac{\pi}{2}+1} x^{\frac{\pi}{2}+1} \bigg) + \frac{1}{1+1} x^{1+1}\\
& \hspace{0.75cm} -7x + 2\ln(|x|) - 6 \bigg( \frac{1}{-3+1} x^{-3+1} \bigg) + c\\
&= \frac{1}{2}x^6 -\frac{2}{\frac{\pi}{2}+1} x^{\frac{\pi}{2}+1}+ \frac{1}{2}x^2 -7x + 2\ln(|x|) +3x^{-2} + c
\end{align*}

\end{mdframed}
\newpage

\begin{enumerate}

\item Find $\displaystyle{ \int 7x^3 \text{ }dx}$
\vspace{4cm} 

\item Find $\displaystyle{ \int 5x^2 - 3 \text{ }dx}$
\vspace{4cm} 
\item Find $\displaystyle{ \int x^2 -2x + 1 \text{ }dx}$
\vspace{4cm} 

\item Find $\displaystyle{ \int -3x^4 +2x^3 - \frac{1}{x} \text{ }dx}$
\vspace{4cm} 
\newpage
\item Find $\displaystyle{ \int x^{-2} -7x^{-3} \text{ }dx}$
\vspace{4cm} 
\item Find $\displaystyle{ \int 7x^6 + x^{\frac{3}{2}} -2 + \frac{2}{x^3} \text{ }dx}$
\vspace{4cm} 
\item Find $\displaystyle{ \int -x^{99} -\frac{50}{x} + 7x^{\pi} - \frac{1}{x^{101}} \text{ }dx}$
\vspace{4cm} 
\item Find $\displaystyle{ \int 7\sqrt{x} + \sqrt[3]{x} - \frac{1}{x^{\frac{4}{5}}}\text{ }dx}$
\vspace{4cm} 
\end{enumerate}
\end{document}