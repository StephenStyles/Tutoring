\documentclass[16pt]{article}

\usepackage{amsmath,amssymb,amsthm,amsfonts,amscd}
\usepackage{showlabels}
\usepackage{color}
\usepackage{hyperref}
\usepackage[numeric]{amsrefs}
\usepackage{graphicx}

\usepackage[framemethod=TikZ]{mdframed}
%\usepackage[framemethod=default]{mdframed}


%\mdfdefinestyle{box}{%
%rightline=true,
%innerleftmargin=10,
%innerrightmargin=10,
%frametitlerule=true,
%frametitlerulecolor=blue,
%frametitlebackgroundcolor=white,
%frametitlerulewidth=2pt}

\mdfdefinestyle{TheoremFrame}{%
    linecolor=blue,
    outerlinewidth=1,
    roundcorner=15,
    innertopmargin= \baselineskip,
    innerbottommargin= \baselineskip,
    innerrightmargin=10,
    innerleftmargin=10,
    backgroundcolor=white}
    
\mdfdefinestyle{ProofFrame}{%
    linecolor=red,
    outerlinewidth=1,
    roundcorner=15,
    innertopmargin= \baselineskip,
    innerbottommargin= \baselineskip,
    innerrightmargin=10,
    innerleftmargin=10,
    backgroundcolor=white}
    
    
\setlength{\oddsidemargin}{0cm}
\setlength{\evensidemargin}{0cm}
\setlength{\marginparwidth}{0in}
\setlength{\marginparsep}{0in}
\setlength{\marginparpush}{0in}
\setlength{\topmargin}{0in}
\setlength{\headheight}{0pt}
\setlength{\headsep}{0pt}
\setlength{\footskip}{.3in}
\setlength{\textheight}{9.2in}
\setlength{\textwidth}{6.0in}
\setlength{\parskip}{0.25pt}
\setlength{\parindent}{0.25in}

    
\newlength\tindent
\setlength{\tindent}{\parindent}
\setlength{\parindent}{0pt}
\renewcommand{\indent}{\hspace*{\tindent}}

\newtheorem*{nthm}{Theorem}
\newtheorem*{nlem}{Lemma}
\newtheorem*{nprop}{Proposition}
\newtheorem*{ncor}{Corollary}
\newtheorem*{nconj}{Conjecture}
\newtheorem*{nclaim}{Claim}
\theoremstyle{remark}
\newtheorem*{define}{Definition}
\newtheorem*{nrem}{Remarks}
\newtheorem*{notation}{Notation}
\newtheorem*{note}{Note}
\newtheorem*{ex}{Example}
\newtheorem*{imt}{Important}
\newtheorem*{fact}{Fact}
\newcommand{\vs}{\vspace{0.1in}}
\newcommand{\Lim}[1]{\raisebox{0.5ex}{\scalebox{0.8}{$\displaystyle \lim_{#1}\;$}}}


\title{Trig Substitution}

\author{Stephen Styles}


\begin{document}

\maketitle

For integrals of the form $\displaystyle{\int (a^2 -b^2x^2)^n \,dx}$ use $x=\frac{a\sin(\theta)}{b}$ as your substitution.
\begin{enumerate}

\item $\displaystyle{\int \sqrt{4-25x^2} dx}$
\vspace{8cm}

\item $\displaystyle{\int \frac{1}{\sqrt{1-9x^2}} dx}$
\vspace{8cm}

\item $\displaystyle{\int \frac{1}{x^4\sqrt{9-x^2}} dx}$
\vspace{9cm}
\item $\displaystyle{\int \frac{\sqrt{16-x^2}}{x^2} dx}$
\vspace{4cm}
\newpage

\end{enumerate}

For integrals of the form $\displaystyle{\int (a^2 +b^2x^2)^n \,dx}$ use $x=\frac{a\tan(\theta)}{b}$ as your substitution.
\begin{enumerate}


\item $\displaystyle{\int \frac{\sqrt{9+x^2}}{x^4} dx}$
\vspace{8cm}

\item $\displaystyle{\int \frac{3x}{x^2+10x+29} dx}$
\vspace{4cm}
\newpage
\item $\displaystyle{\int_{-1}^{1} \frac{1}{(1+x^2)^2} dx}$
\vspace{8cm}

\item $\displaystyle{\int \frac{1}{\sqrt{25x^2+16}} dx}$
\vspace{4cm}
\newpage
\end{enumerate}
For integrals of the form $\displaystyle{\int (b^2x^2 -a^2)^n \,dx}$ use $x=\frac{a\sec(\theta)}{b}$ as your substitution.
\begin{enumerate}

\item $\displaystyle{\int \frac{1}{\sqrt{25x^2-1}} dx}$
\vspace{9cm}
\item $\displaystyle{\int \frac{\sqrt{16x^2-9}}{x} dx}$
\vspace{4cm}
\newpage
\item $\displaystyle{\int \frac{1}{x^2\sqrt{x^2-36}} dx}$
\vspace{9cm}

\item $\displaystyle{\int \frac{x}{\sqrt{x^2-4}} dx}$
\vspace{4cm}


\end{enumerate}


\end{document}