\documentclass[16pt]{article}

\usepackage{amsmath,amssymb,amsthm,amsfonts,amscd}
\usepackage{showlabels}
\usepackage{color}
\usepackage{hyperref}
\usepackage[numeric]{amsrefs}
\usepackage{graphicx}

\usepackage[framemethod=TikZ]{mdframed}
%\usepackage[framemethod=default]{mdframed}


%\mdfdefinestyle{box}{%
%rightline=true,
%innerleftmargin=10,
%innerrightmargin=10,
%frametitlerule=true,
%frametitlerulecolor=blue,
%frametitlebackgroundcolor=white,
%frametitlerulewidth=2pt}

\mdfdefinestyle{TheoremFrame}{%
    linecolor=blue,
    outerlinewidth=1,
    roundcorner=15,
    innertopmargin= \baselineskip,
    innerbottommargin= \baselineskip,
    innerrightmargin=10,
    innerleftmargin=10,
    backgroundcolor=white}
    
\mdfdefinestyle{ProofFrame}{%
    linecolor=red,
    outerlinewidth=1,
    roundcorner=15,
    innertopmargin= \baselineskip,
    innerbottommargin= \baselineskip,
    innerrightmargin=10,
    innerleftmargin=10,
    backgroundcolor=white}
    
    
\setlength{\oddsidemargin}{0cm}
\setlength{\evensidemargin}{0cm}
\setlength{\marginparwidth}{0in}
\setlength{\marginparsep}{0in}
\setlength{\marginparpush}{0in}
\setlength{\topmargin}{0in}
\setlength{\headheight}{0pt}
\setlength{\headsep}{0pt}
\setlength{\footskip}{.3in}
\setlength{\textheight}{9.2in}
\setlength{\textwidth}{6.0in}
\setlength{\parskip}{0.25pt}
\setlength{\parindent}{0.25in}

    
\newlength\tindent
\setlength{\tindent}{\parindent}
\setlength{\parindent}{0pt}
\renewcommand{\indent}{\hspace*{\tindent}}

\newtheorem*{nthm}{Theorem}
\newtheorem*{nlem}{Lemma}
\newtheorem*{nprop}{Proposition}
\newtheorem*{ncor}{Corollary}
\newtheorem*{nconj}{Conjecture}
\newtheorem*{nclaim}{Claim}
\theoremstyle{remark}
\newtheorem*{define}{Definition}
\newtheorem*{nrem}{Remarks}
\newtheorem*{notation}{Notation}
\newtheorem*{note}{Note}
\newtheorem*{ex}{Example}
\newtheorem*{imt}{Important}
\newtheorem*{fact}{Fact}
\newcommand{\vs}{\vspace{0.1in}}
\newcommand{\Lim}[1]{\raisebox{0.5ex}{\scalebox{0.8}{$\displaystyle \lim_{#1}\;$}}}


\title{Integral of $\sec^3(x)$}

\author{Stephen Styles}


\begin{document}

\maketitle

\begin{align*}
\int \sec^3 (x)\, dx &= \int \sec(x) \sec^2(x) \, dx\\
\end{align*}
Let $u= \sec(x)$, then $du = \sec(x)\tan(x)$\\

Let $dv = \sec^2(x) \, dx$, then $v = \tan(x)$

\begin{align*}
\int \sec^3 (x)\, dx &= uv - \int v\, du\\
&= \sec(x)\tan(x) - \int \tan(x) \sec(x) \tan(x) \, dx\\
&= \sec(x)\tan(x) - \int \tan^2(x) \sec(x) \, dx\\
&= \sec(x)\tan(x) - \int (\sec^2(x)-1) \sec(x) \, dx\\
&= \sec(x)\tan(x) - \int \big(\sec^3(x)-\sec(x)\big) \, dx\\
&= \sec(x)\tan(x) - \int \sec^3(x)\, dx + \int \sec(x) \, dx\\
\end{align*}
Adding $\displaystyle{ \int \sec^3(x) \, dx}$ to both sides we get:
\begin{align*}
2 \int \sec^3 (x)\, dx &= \sec(x)\tan(x)+ \int \sec(x) \, dx\\
&= \sec(x)\tan(x) + \ln|\sec(x)+\tan(x)| + C
\end{align*}
Therefore:
\begin{align*}
\int \sec^3 (x)\, dx &= \frac{\sec(x)\tan(x) + \ln|\sec(x)+\tan(x)|}{2} + C
\end{align*}

\end{document}