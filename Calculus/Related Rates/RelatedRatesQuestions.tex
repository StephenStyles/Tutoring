\documentclass[12pt]{article}

\usepackage{bm,amsthm,amssymb,amsfonts,mathtools}
\usepackage[margin=1in]{geometry} 
\usepackage{dsfont}
% neural net drawing
\usepackage{tikz}
\usetikzlibrary{positioning}

\usepackage{neuralnetwork}

% macros
\newcommand{\norm}[1]{\left\lVert#1\right\rVert}

\usepackage{amsmath,amssymb,amsthm,amsfonts,amscd}
\usepackage{showlabels}
\usepackage{color}
\usepackage{hyperref}
\usepackage[numeric]{amsrefs}
\usepackage{graphicx}

\usepackage[framemethod=TikZ]{mdframed}
%\usepackage[framemethod=default]{mdframed}


%\mdfdefinestyle{box}{%
%rightline=true,
%innerleftmargin=10,
%innerrightmargin=10,
%frametitlerule=true,
%frametitlerulecolor=blue,
%frametitlebackgroundcolor=white,
%frametitlerulewidth=2pt}

\mdfdefinestyle{TheoremFrame}{%
    linecolor=blue,
    outerlinewidth=1,
    roundcorner=15,
    innertopmargin= \baselineskip,
    innerbottommargin= \baselineskip,
    innerrightmargin=10,
    innerleftmargin=10,
    backgroundcolor=white}
    
\mdfdefinestyle{ProofFrame}{%
    linecolor=red,
    outerlinewidth=1,
    roundcorner=15,
    innertopmargin= \baselineskip,
    innerbottommargin= \baselineskip,
    innerrightmargin=10,
    innerleftmargin=10,
    backgroundcolor=white}
    
    
\setlength{\oddsidemargin}{0cm}
\setlength{\evensidemargin}{0cm}
\setlength{\marginparwidth}{0in}
\setlength{\marginparsep}{0in}
\setlength{\marginparpush}{0in}
\setlength{\topmargin}{0in}
\setlength{\headheight}{0pt}
\setlength{\headsep}{0pt}
\setlength{\footskip}{.3in}
\setlength{\textheight}{9.2in}
\setlength{\textwidth}{6.0in}
\setlength{\parskip}{0.25pt}
\setlength{\parindent}{0.25in}

    
\newlength\tindent
\setlength{\tindent}{\parindent}
\setlength{\parindent}{0pt}
\renewcommand{\indent}{\hspace*{\tindent}}

\newtheorem*{nthm}{Theorem}
\newtheorem*{nlem}{Lemma}
\newtheorem*{nprop}{Proposition}
\newtheorem*{ncor}{Corollary}
\newtheorem*{nconj}{Conjecture}
\newtheorem*{nclaim}{Claim}
\theoremstyle{remark}
\newtheorem*{define}{Definition}
\newtheorem*{nrem}{Remarks}
\newtheorem*{notation}{Notation}
\newtheorem*{note}{Note}
\newtheorem*{ex}{Example}
\newtheorem*{imt}{Important}
\newtheorem*{fact}{Fact}
\newcommand{\vs}{\vspace{0.1in}}
\newcommand{\Lim}[1]{\raisebox{0.5ex}{\scalebox{0.8}{$\displaystyle \lim_{#1}\;$}}}
\title{Related Rates}
\author{Stephen Styles}
\begin{document}
\maketitle


\begin{enumerate}
\item A dump-truck is pouring sand onto a pile at a rate of $0.5m^3/\text{min}$. The pile of sand is the shape of a cone whose base is always the same length as the height. How fast is the pile increasing when the height is $4\text{cm}$.
\vspace{7cm}
\item A baseball player is currently on first base. He starts running from first base to second base at a rate of $30\text{feet}/\text{sec}$. How fast is the distance between him and homeplate changing when the player is $45$ feet off first base?
\newpage
\item A spherical snowball melts at a rate of $2\pi \text{cm}^3/\text{hr}$. It melts symmetrically such that it is always a sphere. How fast is its radius changing at the instant $r=10$.
\vspace{8cm}
\item A tank of water the shape of a cone is being drained with water at a rate of $15m^3/\text{min}$. The base radius is $27$ meters and the height is $9$ meters. At what rate is the depth of water changing when the radius of the top of the water is $10$ meters.
\newpage
\item The angle of elevation is the angle formed by a horizontal line and a line joining the observers eye to an object above the horizontal line. A person is $5$km away from a launch pad for a rocket. The rocket is blasting off at a rate of $8$km/s. At what rate is the angle of elevation $\theta$, changing when the rocket is $40$km above the ground.
\end{enumerate}
\end{document}