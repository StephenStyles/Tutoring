\documentclass[16pt]{article}

\usepackage{amsmath,amssymb,amsthm,amsfonts,amscd}
\usepackage{showlabels}
\usepackage{color}
\usepackage{hyperref}
\usepackage[numeric]{amsrefs}
\usepackage{graphicx}

\usepackage[framemethod=TikZ]{mdframed}
%\usepackage[framemethod=default]{mdframed}


%\mdfdefinestyle{box}{%
%rightline=true,
%innerleftmargin=10,
%innerrightmargin=10,
%frametitlerule=true,
%frametitlerulecolor=blue,
%frametitlebackgroundcolor=white,
%frametitlerulewidth=2pt}

\mdfdefinestyle{TheoremFrame}{%
    linecolor=blue,
    outerlinewidth=1,
    roundcorner=15,
    innertopmargin= \baselineskip,
    innerbottommargin= \baselineskip,
    innerrightmargin=10,
    innerleftmargin=10,
    backgroundcolor=white}
    
\mdfdefinestyle{ProofFrame}{%
    linecolor=red,
    outerlinewidth=1,
    roundcorner=15,
    innertopmargin= \baselineskip,
    innerbottommargin= \baselineskip,
    innerrightmargin=10,
    innerleftmargin=10,
    backgroundcolor=white}
    
    
\setlength{\oddsidemargin}{0cm}
\setlength{\evensidemargin}{0cm}
\setlength{\marginparwidth}{0in}
\setlength{\marginparsep}{0in}
\setlength{\marginparpush}{0in}
\setlength{\topmargin}{0in}
\setlength{\headheight}{0pt}
\setlength{\headsep}{0pt}
\setlength{\footskip}{.3in}
\setlength{\textheight}{9.2in}
\setlength{\textwidth}{6.0in}
\setlength{\parskip}{0.25pt}
\setlength{\parindent}{0.25in}
\allowdisplaybreaks
    
\newlength\tindent
\setlength{\tindent}{\parindent}
\setlength{\parindent}{0pt}
\renewcommand{\indent}{\hspace*{\tindent}}

\newtheorem*{nthm}{Theorem}
\newtheorem*{nlem}{Lemma}
\newtheorem*{nprop}{Proposition}
\newtheorem*{ncor}{Corollary}
\newtheorem*{nconj}{Conjecture}
\newtheorem*{nclaim}{Claim}
\theoremstyle{remark}
\newtheorem*{define}{Definition}
\newtheorem*{nrem}{Remarks}
\newtheorem*{notation}{Notation}
\newtheorem*{note}{Note}
\newtheorem*{ex}{Example}
\newtheorem*{imt}{Important}
\newtheorem*{fact}{Fact}
\newcommand{\vs}{\vspace{0.1in}}
\newcommand{\Lim}[1]{\raisebox{0.5ex}{\scalebox{0.8}{$\displaystyle \lim_{#1}\;$}}}


\title{Surface Area of Rotations}

\author{Stephen Styles}


\begin{document}

\maketitle

Let $y = f(x) \geq 0$, if $f(x)$ is a smooth (continuously differentiable) function on the interval ${[a,b]}$, then the surface area generated by revolving the function about the line $y=r$ can be calculated by
\begin{align*}
S &= \int_a^b 2\pi (f(x)-r) \sqrt{1+ \big( f'(x) \big)^2} \, dx.
\end{align*}

Similarly, if $x = g(y) \geq 0$, where $g(y)$ is a smooth function on ${[c,d]}$, then the surface area generated by revolving the function around the line $x=r$ can be calculated by
\begin{align*}
S &= \int_c^d 2\pi (g(y)-r) \sqrt{1+\big( g'(y) \big)^2} \, dy.
\end{align*}

Questions:
\begin{enumerate}
\item Find the surface area of the solid created by the rotating the function $f(x) = \sqrt{16-x^2}$ around the $x$-axis for $-2 \leq x \leq 2$.
\newpage
\item Find the surface area of the solid created by the rotating the function $f(x) = x^3$ around the $x$-axis for $0 \leq x \leq 1$
\newpage
\item Find the surface area of the solid created by the rotating the function $9x = y^2+18$ around the $x$-axis for $2 \leq x \leq 6$.
\newpage
\item Find the surface area of the solid created by the rotating the function $x = e^{-2y}$ around the $y$-axis for $0 \leq x \leq 1$.
\newpage
\item Find the surface area of the solid created by the rotating the function $f(x) = \frac{1}{3}x$ around the $x$-axis for $1 \leq x \leq 3$.
\newpage
\item Find the surface area of the solid created by the rotating the function $f(x) = \sqrt{x}$ around the $x$-axis for $2 \leq x \leq 6$.
\newpage
\item Find the surface area generated by revolving the curve $y=\frac{x^3}{3}+\frac{1}{4x}$, $1\leq x \leq 3$ about the line $y=-2$.
\newpage
\end{enumerate}

\end{document}