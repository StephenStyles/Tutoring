\documentclass[16pt]{article}

\usepackage{amsmath,amssymb,amsthm,amsfonts,amscd}
\usepackage{showlabels}
\usepackage{color}
\usepackage{hyperref}
\usepackage[numeric]{amsrefs}
\usepackage{graphicx}

\usepackage[framemethod=TikZ]{mdframed}
%\usepackage[framemethod=default]{mdframed}


%\mdfdefinestyle{box}{%
%rightline=true,
%innerleftmargin=10,
%innerrightmargin=10,
%frametitlerule=true,
%frametitlerulecolor=blue,
%frametitlebackgroundcolor=white,
%frametitlerulewidth=2pt}

\mdfdefinestyle{TheoremFrame}{%
    linecolor=blue,
    outerlinewidth=1,
    roundcorner=15,
    innertopmargin= \baselineskip,
    innerbottommargin= \baselineskip,
    innerrightmargin=10,
    innerleftmargin=10,
    backgroundcolor=white}
    
\mdfdefinestyle{ProofFrame}{%
    linecolor=red,
    outerlinewidth=1,
    roundcorner=15,
    innertopmargin= \baselineskip,
    innerbottommargin= \baselineskip,
    innerrightmargin=10,
    innerleftmargin=10,
    backgroundcolor=white}
    
    
\setlength{\oddsidemargin}{0cm}
\setlength{\evensidemargin}{0cm}
\setlength{\marginparwidth}{0in}
\setlength{\marginparsep}{0in}
\setlength{\marginparpush}{0in}
\setlength{\topmargin}{0in}
\setlength{\headheight}{0pt}
\setlength{\headsep}{0pt}
\setlength{\footskip}{.3in}
\setlength{\textheight}{9.2in}
\setlength{\textwidth}{6.0in}
\setlength{\parskip}{0.25pt}
\setlength{\parindent}{0.25in}
\allowdisplaybreaks
    
\newlength\tindent
\setlength{\tindent}{\parindent}
\setlength{\parindent}{0pt}
\renewcommand{\indent}{\hspace*{\tindent}}

\newtheorem*{nthm}{Theorem}
\newtheorem*{nlem}{Lemma}
\newtheorem*{nprop}{Proposition}
\newtheorem*{ncor}{Corollary}
\newtheorem*{nconj}{Conjecture}
\newtheorem*{nclaim}{Claim}
\theoremstyle{remark}
\newtheorem*{define}{Definition}
\newtheorem*{nrem}{Remarks}
\newtheorem*{notation}{Notation}
\newtheorem*{note}{Note}
\newtheorem*{ex}{Example}
\newtheorem*{imt}{Important}
\newtheorem*{fact}{Fact}
\newcommand{\vs}{\vspace{0.1in}}
\newcommand{\Lim}[1]{\raisebox{0.5ex}{\scalebox{0.8}{$\displaystyle \lim_{#1}\;$}}}


\title{Surface Area of Rotations}

\author{Stephen Styles}


\begin{document}

\maketitle

Let $y = f(x) \geq 0$, if $f(x)$ is a smooth (continuously differentiable) function on the interval ${[a,b]}$, then the surface area generated by revolving the function about the line $y=r$ can be calculated by
\begin{align*}
S &= \int_a^b 2\pi (f(x)-r) \sqrt{1+ \big( f'(x) \big)^2} \, dx.
\end{align*}

Similarly, if $x = g(y) \geq 0$, where $g(y)$ is a smooth function on ${[c,d]}$, then the surface area generated by revolving the function around the line $x=r$ can be calculated by
\begin{align*}
S &= \int_c^d 2\pi (g(y)-r) \sqrt{1+\big( g'(y) \big)^2} \, dy.
\end{align*}

Examples:
\begin{enumerate}
\item Find the surface area of the solid created by the rotating the function $f(x) = \sqrt{x}$ around the $x$-axis for $2 \leq x \leq 6$.
\begin{mdframed}[style=TheoremFrame]
\textit{Solution:}
\begin{align*}
f'(x) &= \frac{1}{2\sqrt{x}}
\end{align*}

\begin{align*}
S&= \int_2^6 2\pi \sqrt{x} \sqrt{1 + \bigg(\frac{1}{2\sqrt{x}}\bigg)^2} \, dx\\[1.5ex]
&= \int_2^6 2\pi \sqrt{x} \sqrt{1 + \frac{1}{4x}} \, dx\\[1.5ex]
&= \int_2^6 2\pi \sqrt{x} \sqrt{\frac{4x+1}{4x}} \, dx\\[1.5ex]
&= \int_2^6 \pi \sqrt{4x+1} \, dx\\[1.5ex]
&= \frac{\pi}{6} (4x+1)^{3/2} \bigg|_2^6\\[1.5ex]
&= \frac{\pi}{6}\bigg(125-27)\\[1.5ex]
&= \frac{49\pi}{3} 
\end{align*}
\end{mdframed}
\newpage
\item Find the surface area of the solid created by the rotating the function $f(x) = \sqrt{16-x^2}$ around the $x$-axis for $-1 \leq x \leq 1$.
\begin{mdframed}[style=TheoremFrame]
\textit{Solution:}
\begin{align*}
f'(x) &= \frac{1}{\sqrt{16-x^2}}(-2x)\\[1.1ex]
&= \frac{-2x}{\sqrt{16-x^2}}
\end{align*}
\vspace*{-0.5cm}
\begin{align*}
S&= \int_{-1}^1 2\pi \sqrt{16-x^2} \sqrt{1+\bigg(\frac{-2x}{\sqrt{16-x^2}}\bigg)^2}\, dx\\[1.1ex]
&= \int_{-1}^1 2\pi \sqrt{16-x^2} \sqrt{1+\frac{4x^2}{16-x^2}}\, dx\\[1.1ex]
&= \int_{-1}^1 2\pi \sqrt{16-x^2} \sqrt{\frac{16-x^2+4x^2}{16-x^2}}\, dx\\[1.1ex]
&= \int_{-1}^1 2\pi \sqrt{16+3x^2}\, dx\\[1.1ex]
&= 4\pi \int_0^1 \sqrt{16+3x^2}\, dx
\end{align*}
Let $\displaystyle{x=\frac{4}{\sqrt{3}} \tan(\theta)}$, then $\displaystyle{dx = \frac{4}{\sqrt{3}} \sec^2(\theta)\,d\theta}$.
\begin{align*}
S&= 4\pi \int_a^b \sqrt{16+3\bigg(\frac{4}{\sqrt{3}}\tan(\theta)\bigg)^2}\frac{4}{\sqrt{3}} \sec^2(\theta)\, d\theta\\[1.1ex]
&= \frac{16\pi}{\sqrt{3}} \int_a^b \sqrt{16+16\tan^2(\theta)} \sec^2(\theta) \, d\theta\\[1.1ex]
&= \frac{16\pi}{\sqrt{3}} \int_a^b \sqrt{16\sec^2(\theta)} \sec^2(\theta) \, d\theta\\[1.1ex]
&= \frac{16\pi}{\sqrt{3}} \int_a^b 4\sec(\theta) \sec^2(\theta) \, d\theta\\[1.1ex]
&= \frac{64\pi}{\sqrt{3}}\int_a^b  \sec^3(\theta) \, d\theta\\[1.1ex]
&= \frac{64\pi}{\sqrt{3}} \bigg(\frac{\sec(\theta)\tan(\theta) + \ln\big|\sec(\theta)+\tan(\theta)\big|}{2}\bigg) \bigg|_a^b\\[1.1ex]
&= \frac{64\pi}{\sqrt{3}} \bigg(\frac{\frac{\sqrt{16+3x^2}}{4}\frac{\sqrt{3}x}{4} + \ln\big|\frac{\sqrt{16+3x^2}}{4}+\frac{\sqrt{3}x}{4}\big|}{2}\bigg) \bigg|_0^1\\[1.1ex]
&= \bigg(2\pi x \sqrt{16+3x^2} + \frac{32\pi}{\sqrt{3}} \ln\big|\frac{\sqrt{16+3x^2}}{4}+\frac{\sqrt{3}x}{4}\big|\bigg)_0^1\\[1.1ex]
&= 2\pi \sqrt{19} + \frac{32\pi}{\sqrt{3}} \ln\bigg(\frac{\sqrt{19}+\sqrt{3}}{4}\bigg) 
\end{align*}
\end{mdframed}
\end{enumerate}
\newpage
Questions:
\begin{enumerate}

\item Find the surface area of the solid created by the rotating the function $f(x) = x^3$ around the $x$-axis for $0 \leq x \leq 1$
\begin{mdframed}[style=TheoremFrame]
\textit{Solution:}
\begin{align*}
f'(x) &= 3x^2
\end{align*}
\begin{align*}
S&= \int_0^1 2\pi x^3 \sqrt{1+\big(3x^2\big)^2} \, dx\\[1.5ex]
&= \int_0^1 2\pi x^3 \sqrt{1+9x^4} \, dx
\end{align*}
Let $u = 1+9x^4$, then $du = 36x^3 \, dx$
\begin{align*}
S&= \frac{\pi}{18} \int_a^b \sqrt{u} \, du\\[1.5ex]
&= \frac{\pi}{18} \frac{2}{3} u^{3/2} \bigg|_a^b\\[1.5ex]
&= \frac{\pi}{27}  (1+9x^4)^{3/2} \bigg|_0^1\\[1.5ex]
&= \frac{\pi}{27} \big( 10^{3/2} - 1 \big)
\end{align*}
\end{mdframed}
\newpage
\item Find the surface area of the solid created by the rotating the function $9y = x^2+18$ around the $y$-axis for $12 \leq y \leq 20$.
\begin{mdframed}[style=TheoremFrame]
\textit{Solution:}
\begin{align*}
x^2+18&=9y\\
\Rightarrow x^2 &= 9y-18\\[1.5ex]
\Rightarrow x &= \sqrt{9y-18}\\[1.5ex]
&= 3 \sqrt{y-2}
\end{align*}
\begin{align*}
g'(y) &= \frac{3}{2\sqrt{y-2}}
\end{align*}
\begin{align*}
S &= \int_{12}^{20} 2 \pi \cdot 3 \sqrt{y-2} \sqrt{1 + \bigg(\frac{3}{2\sqrt{y-2}}\bigg)^2} \, dy\\[1.5ex]
&= 6\pi \int_{12}^{20} \sqrt{y-2}  \sqrt{1 +\frac{9}{4(y-2)}} \, dy\\[1.5ex]
&= 6\pi \int_{12}^{20} \sqrt{y-2}  \sqrt{\frac{4(y-2)+9}{4(y-2)}} \, dy\\[1.5ex]
&= 6\pi \int_{12}^{20} \sqrt{4(y-2)+9} \, dy\\[1.5ex]
&= 6\pi \int_{12}^{20} \sqrt{4y+1} \, dy\\[1.5ex]
&= 6\pi \cdot \frac{2}{3} (4y+1)^{3/2} \bigg|_{12}^{20}\\[1.5ex]
&= 4\pi (729-343)\\[1.5ex]
&= 1544\pi
\end{align*}
\end{mdframed}
\newpage
\item Find the surface area of the solid created by the rotating the function $f(x) = \frac{1}{3}x$ around the $x$-axis for $1 \leq x \leq 3$.
\begin{mdframed}[style=TheoremFrame]
\textit{Solution:}
\begin{align*}
f'(x) &= \frac{1}{3}
\end{align*}
\begin{align*}
S &= \int_1^3 2 \pi \frac{1}{3} x \sqrt{1+\bigg(\frac{1}{3}\bigg)^2} \, dx\\[1.5ex]
&= \frac{2\pi}{3} \int_1^3 x \sqrt{1 + \frac{1}{9}} \, dx\\[1.5ex]
&= \frac{2\pi}{3} \sqrt{\frac{10}{9}} \int_1^3 x \, dx\\[1.5ex]
&= \frac{\pi}{3} \sqrt{\frac{10}{9}} x^2 \bigg|_1^3\\[1.5ex]
&= \frac{8\sqrt{10}\pi}{9}
\end{align*}
\end{mdframed}
\newpage
\item Find the surface area generated by revolving the curve $\displaystyle{y=\frac{x^3}{3}+\frac{1}{4x}}$, $1\leq x \leq 3$ about the line $y=-2$.
\begin{mdframed}[style=TheoremFrame]
\textit{Solution:}
\begin{align*}
f'(x) &= x^2-\frac{1}{4x^2}
\end{align*}
\begin{align*}
S&= \int_1^3 2\pi \bigg(\frac{x^3}{3}+\frac{1}{4x}+2 \bigg) \sqrt{1 + \bigg(x^2-\frac{1}{4x^2}\bigg)^2} \, dx\\[1.5ex]
&= 2\pi \int_1^3 \bigg(\frac{x^3}{3}+\frac{1}{4x} +2\bigg) \sqrt{1 + \big(x^2\big)^2-\frac{1}{2}+\bigg(\frac{1}{4x^2}\bigg)^2} \, dx\\[1.5ex]
&= 2\pi \int_1^3 \bigg(\frac{x^3}{3}+\frac{1}{4x} +2\bigg) \sqrt{\big(x^2\big)^2+\frac{1}{2}+\bigg(\frac{1}{4x^2}\bigg)^2} \, dx\\[1.5ex]
&= 2\pi \int_1^3 \bigg(\frac{x^3}{3}+\frac{1}{4x} +2\bigg) \sqrt{\bigg(x^2+\frac{1}{4x^2}\bigg)^2} \, dx\\[1.5ex]
&= 2\pi \int_1^3 \bigg(\frac{x^3}{3}+\frac{1}{4x} +2\bigg) \bigg(x^2+\frac{1}{4x^2}\bigg) \, dx\\[1.5ex]
&= 2\pi \int_1^3 \frac{x^5}{3} +  \frac{x}{12} + \frac{x}{4} + \frac{1}{16x^3} + 2x^2 + \frac{1}{2x^2} \, dx\\[1.5ex]
&= 2\pi \int_1^3 \frac{x^5}{3} + 2x^2 + \frac{x}{3} +\frac{1}{2x^2}+ \frac{1}{16x^3} \, dx\\[1.5ex]
&= 2\pi \bigg(\frac{x^6}{18} + \frac{2x^3}{3}+\frac{x^2}{6} -\frac{1}{2x}- \frac{1}{32x^2}\bigg) \bigg|_1^3\\[1.5ex]
&= \frac{2141\pi}{18}
\end{align*}
\end{mdframed}
\newpage
\item Find the surface area of the solid created by the rotating the function $x = e^{-2y}$ around the $y$-axis for $0 \leq x \leq 1$. (Trig Sub Question. After you solve the integral, do not try to simplify after plugging in the values of the integral)
\begin{mdframed}[style=TheoremFrame]
\textit{Solution:}
\begin{align*}
g'(y) &= -2e^{-2y}
\end{align*}
\begin{align*}
S&= \int_0^1 2\pi e^{-2y} \sqrt{1+\big(-2e^{-2y}\big)^2} \, dy\\
\end{align*}
Let $\displaystyle{e^{-2y} = \frac{1}{2}\tan(\theta)}$, then $\displaystyle{-2e^{-2y} \, dy = \frac{1}{2} \sec^2(\theta) \, d\theta}$
\begin{align*}
S&= \frac{-\pi}{2} \int_a^b \sqrt{1+\tan^2(\theta)} \sec^2 (\theta) \, d\theta\\[1.5ex]
&= \frac{-\pi}{2} \int_a^b \sqrt{\sec^2(\theta)} \sec^2 (\theta) \, d\theta\\[1.5ex]
&= \frac{-\pi}{2} \int_a^b \sec^3 (\theta) \, d\theta\\[1.5ex]
&= \frac{-\pi}{2}\bigg(\frac{\sec(\theta)\tan(\theta) + \ln\big|\sec(\theta)+\tan(\theta)\big|}{2}\bigg) \bigg|_a^b\\[1.5ex]
&= \frac{-\pi}{2}\bigg(\frac{\sqrt{1+4e^{-4y}}e^{-2y} + \ln\big|\sqrt{1+4e^{-4y}}+e^{-2y}\big|}{2}\bigg) \bigg|_0^1\\[1.5ex]
&\approx 4.21593
\end{align*}
\end{mdframed}
\end{enumerate}

\end{document}