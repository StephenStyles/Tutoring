\documentclass[16pt]{article}

\usepackage{array,amsmath,amssymb,amsthm,amsfonts,amscd}
\usepackage{showlabels}
\usepackage{color}
\usepackage{hyperref}
\usepackage[numeric]{amsrefs}
\usepackage{graphicx}

\usepackage[framemethod=TikZ]{mdframed}
%\usepackage[framemethod=default]{mdframed}


%\mdfdefinestyle{box}{%
%rightline=true,
%innerleftmargin=10,
%innerrightmargin=10,
%frametitlerule=true,
%frametitlerulecolor=blue,
%frametitlebackgroundcolor=white,
%frametitlerulewidth=2pt}
\mdfdefinestyle{TheoremFrame}{%
    linecolor=blue,
    outerlinewidth=1,
    roundcorner=15,
    innertopmargin= \baselineskip,
    innerbottommargin= \baselineskip,
    innerrightmargin=10,
    innerleftmargin=10,
    backgroundcolor=white}
    
\mdfdefinestyle{ProofFrame}{%
    linecolor=red,
    outerlinewidth=1,
    roundcorner=15,
    innertopmargin= \baselineskip,
    innerbottommargin= \baselineskip,
    innerrightmargin=10,
    innerleftmargin=10,
    backgroundcolor=white}
    
    
\setlength{\oddsidemargin}{0cm}
\setlength{\evensidemargin}{0cm}
\setlength{\marginparwidth}{0in}
\setlength{\marginparsep}{0in}
\setlength{\marginparpush}{0in}
\setlength{\topmargin}{0in}
\setlength{\headheight}{0pt}
\setlength{\headsep}{0pt}
\setlength{\footskip}{.3in}
\setlength{\textheight}{9.2in}
\setlength{\textwidth}{6.0in}
\setlength{\parskip}{0.25pt}
\setlength{\parindent}{0.25in}
\allowdisplaybreaks
    
\newlength\tindent
\setlength{\tindent}{\parindent}
\setlength{\parindent}{0pt}
\renewcommand{\indent}{\hspace*{\tindent}}

\newtheorem*{nthm}{Theorem}
\newtheorem*{nlem}{Lemma}
\newtheorem*{nprop}{Proposition}
\newtheorem*{ncor}{Corollary}
\newtheorem*{nconj}{Conjecture}
\newtheorem*{nclaim}{Claim}
\theoremstyle{remark}
\newtheorem*{define}{Definition}
\newtheorem*{nrem}{Remarks}
\newtheorem*{notation}{Notation}
\newtheorem*{note}{Note}
\newtheorem*{ex}{Example}
\newtheorem*{imt}{Important}
\newtheorem*{fact}{Fact}
\newcommand{\vs}{\vspace{0.1in}}
\newcommand{\Lim}[1]{\raisebox{0.5ex}{\scalebox{0.8}{$\displaystyle \lim_{#1}\;$}}}


\title{Arc Length}

\author{Stephen Styles}


\begin{document}

\maketitle
\vspace*{-0.5cm}

Let $y = f(x)$ be a smooth function (twice differentiable) on the interval ${[ a, b ]}$ then the arc length of $f(x)$ on the interval is given by 
\begin{align*}
s = \int_a^b \sqrt{1+{[f'(x)]}^2} \, dx.
\end{align*} 
We can also consider functions $x=g(y)$. Let $g(y)$ be a smooth function on the interval ${[c,d}]$ then the arc length of $g(y)$ is given by
\begin{align*}
s = \int_c^d \sqrt{1+{[g'(y)]}^2} \, dy.
\end{align*}

Note: The general equation for arc length is 
\begin{align*}
s = \int \sqrt{dx^2 + dy^2}.
\end{align*}
Examples:
\begin{enumerate}
\item Find the arc length of $\displaystyle{f(x) = \frac{x^{3/2}}{3}}$ from $0$ to $5$.
\begin{mdframed}[style=TheoremFrame]
\textit{Solution}:
\vspace*{-0.5cm}
\begin{align*}
f'(x) &= \frac{x^{1/2}}{2}
\end{align*}
\vspace*{-0.5cm}
\begin{align*}
s &= \int_0^5 \sqrt{1 + \bigg(\frac{\sqrt{x}}{2}\bigg)^2}\, dx\\
&= \int_0^5 \sqrt{1 + \frac{x}{4}}\, dx\\
&= \int_0^5 \frac{1}{2}(4+x)^{1/2} \, dx\\
&= \frac{1}{3} (4+x)^{3/2} \bigg|_0^5 \\
&= \frac{19}{3}
\end{align*}
\end{mdframed}
\newpage
\item Find the arc length of a function whose derivative is given by $\displaystyle{f'(x) = \frac{1}{2}\bigg(x^2 - \frac{1}{x^2}\bigg)}$\\ from $\displaystyle{\frac{1}{2}}$ to $2$.
\begin{mdframed}[style=TheoremFrame]
\textit{Solution}:
\vspace*{-0.5cm}
\begin{align*}
s &= \int_{1/2}^2 \sqrt{1+\bigg(\frac{x^2}{2}-\frac{1}{2x^2}\bigg)^2}\, dx\\
&= \int_{1/2}^2 \sqrt{1+\frac{x^4}{4}-\frac{1}{2}+\frac{1}{4x^4}}\, dx\\
&= \int_{1/2}^2 \sqrt{\frac{x^4}{4}+\frac{1}{2}+\frac{1}{4x^4}}\, dx\\
&= \int_{1/2}^2 \sqrt{\bigg(\frac{x^2}{2}+\frac{1}{2x^2}\bigg)^2}\, dx\\
&= \int_{1/2}^2 \frac{x^2}{2}+\frac{1}{2x^2} \, dx\\
&= \frac{x^3}{6}-\frac{1}{2x} \bigg|_{1/2}^2\\
&= \frac{33}{16}
\end{align*}
\end{mdframed}

\item  Find the arc length of  $\displaystyle{g(y) = y^2 - \frac{1}{8}\ln(y)}$ from $\displaystyle{1}$ to $2$.
\begin{mdframed}[style=TheoremFrame]
\textit{Solution}:
\vspace*{-0.5cm}
\begin{align*}
g'(y) &= 2y-\frac{1}{8y}
\end{align*}
\vspace*{-0.5cm}
\begin{align*}
s &= \int_1^2 \sqrt{1+\bigg(2y-\frac{1}{8y}\bigg)^2}\, dy\\
&= \int_1^2 \sqrt{1+4y^2-\frac{1}{2}+\frac{1}{64y^2}}\, dy\\
&= \int_1^2 \sqrt{4y^2+\frac{1}{2}+\frac{1}{64y^2}}\, dy\\
&= \int_1^2 \sqrt{\bigg(2y+\frac{1}{8y}\bigg)^2}\, dy\\
&= \int_1^2 2y+\frac{1}{8y}\, dy\\
&= y^2 + \frac{\ln(y)}{8} \bigg|_1^2\\
&= 3+\frac{\ln(2)}{8}
\end{align*}

\end{mdframed}
\newpage
\end{enumerate}

Questions:
\begin{enumerate}
\item Find the arc length of $f(x) = \ln(\cos(x))$ from $0$ to $\displaystyle{\frac{\pi}{4}}$.
\begin{mdframed}[style=TheoremFrame]
\textit{Solution:}
\begin{align*}
f'(x) &= \frac{1}{\cos(x)} \big(-\sin(x)\big)\\
&= -\tan(x)
\end{align*}
\begin{align*}
s&= \int_0^{\frac{\pi}{4}} \sqrt{1+\big(-\tan(x)\big)^2}\,dx\\
&= \int_0^{\frac{\pi}{4}} \sqrt{1+\tan^2(x)}\, dx\\
&= \int_0^{\frac{\pi}{4}} \sqrt{\sec^2(x)}\, dx\\
&= \int_0^{\frac{\pi}{4}} \sec(x) \, dx\\
&= \ln|\sec(x)+\tan(x)| \bigg|_0^{\frac{\pi}{4}}\\
&= \ln(\sqrt{2}+1)
\end{align*}
\end{mdframed}
\item Find the arc length of $f(x) = \cosh(x)$ from $0$ to $3\pi$.
\begin{mdframed}[style=TheoremFrame]
\textit{Solution:}
\begin{align*}
f'(x) &= \sinh(x)
\end{align*}
\begin{align*}
s &= \int_0^{3\pi} \sqrt{1+\sinh^2(x)}\, dx\\
&= \int_0^{3\pi} \sqrt{\cosh^2(x)}\, dx\\
&=\int_0^{3\pi} \cosh(x)\, dx\\
&=\sinh(x) \bigg|_0^{3\pi}\\
&= \sinh(3\pi)
\end{align*}
\end{mdframed}
\newpage
\item Find the arc length of $\displaystyle{f(x) = \frac{2}{3x^{1/3}}}$ from $1$ to $8$.
\begin{mdframed}[style=TheoremFrame]
\textit{Solution:}
\begin{align*}
f'(x) &= \frac{2}{3x^{1/3}}
\end{align*}
\begin{align*}
s&= \int_1^8 \sqrt{1+\bigg(\frac{2}{3x^{1/3}}\bigg)^2}\, dx\\
&= \int_1^8 \sqrt{1+\frac{4}{9x^{2/3}}}\, dx\\
&= \int_1^8 \sqrt{\frac{9x^{2/3}+4}{9x^{2/3}}}\, dx\\
&= \frac{1}{3}\int_1^8 \frac{\sqrt{9x^{2/3}+4}}{x^{1/3}}\, dx\\
\end{align*}
Let $\displaystyle{u=9x^{2/3}+4}$ then $\displaystyle{du =\frac{6}{x^{1/3}} dx}$\\

$9(1)^{2/3} + 4 = 13$, $9(8)^{2/3}+4=40$
\begin{align*}
s &= \frac{1}{18}\int_{13}^{40} u^{1/2} \,du\\
&= \frac{1}{18}\frac{2}{3} u^{3/2} \bigg|_{13}^{40}\\
&= \frac{1}{27}\bigg(40^{3/2}-13^{3/2}\bigg)
\end{align*}
\end{mdframed}
\newpage
\item Find the arc length of $\displaystyle{g(y) = 2y^3 + \frac{1}{24y}}$ from $y=1$ to $y=3$.
\begin{mdframed}[style=TheoremFrame]
\textit{Solution:}
\begin{align*}
g'(y) &= 6y^2 - \frac{1}{24y^2}
\end{align*}
\begin{align*}
s &= \int_1^3 \sqrt{1+\bigg(6y^2-\frac{1}{24y^2}\bigg)^2}\, dy\\
&= \int_1^3 \sqrt{1+\big(6y^2\big)^2-\frac{1}{2}+\bigg(\frac{1}{24y^2}\bigg)^2}\, dy\\
&= \int_1^3 \sqrt{\big(6y^2\big)^2+\frac{1}{2}+\bigg(\frac{1}{24y^2}\bigg)^2}\, dy\\
&= \int_1^3 \sqrt{\bigg(6y^2+\frac{1}{24y^2}\bigg)^2}\, dy\\
&= \int_1^3 6y^2+\frac{1}{24y^2}\, dy\\
&= 2y^3 - \frac{1}{24y} \bigg|_1^3\\
&= \frac{1873}{36}
\end{align*}

\end{mdframed}
\end{enumerate}


\end{document}