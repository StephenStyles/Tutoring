\documentclass[16pt]{article}

\usepackage{amsmath,amssymb,amsthm,amsfonts,amscd}
\usepackage{showlabels}
\usepackage{color}
\usepackage{hyperref}
\usepackage[numeric]{amsrefs}
\usepackage{graphicx}

\usepackage[framemethod=TikZ]{mdframed}
%\usepackage[framemethod=default]{mdframed}


%\mdfdefinestyle{box}{%
%rightline=true,
%innerleftmargin=10,
%innerrightmargin=10,
%frametitlerule=true,
%frametitlerulecolor=blue,
%frametitlebackgroundcolor=white,
%frametitlerulewidth=2pt}

\mdfdefinestyle{TheoremFrame}{%
    linecolor=blue,
    outerlinewidth=1,
    roundcorner=15,
    innertopmargin= \baselineskip,
    innerbottommargin= \baselineskip,
    innerrightmargin=10,
    innerleftmargin=10,
    backgroundcolor=white}
    
\mdfdefinestyle{ProofFrame}{%
    linecolor=red,
    outerlinewidth=1,
    roundcorner=15,
    innertopmargin= \baselineskip,
    innerbottommargin= \baselineskip,
    innerrightmargin=10,
    innerleftmargin=10,
    backgroundcolor=white}
    
    
\setlength{\oddsidemargin}{0cm}
\setlength{\evensidemargin}{0cm}
\setlength{\marginparwidth}{0in}
\setlength{\marginparsep}{0in}
\setlength{\marginparpush}{0in}
\setlength{\topmargin}{0in}
\setlength{\headheight}{0pt}
\setlength{\headsep}{0pt}
\setlength{\footskip}{.3in}
\setlength{\textheight}{9.2in}
\setlength{\textwidth}{6.0in}
\setlength{\parskip}{0.25pt}
\setlength{\parindent}{0.25in}

    
\newlength\tindent
\setlength{\tindent}{\parindent}
\setlength{\parindent}{0pt}
\renewcommand{\indent}{\hspace*{\tindent}}

\newtheorem*{nthm}{Theorem}
\newtheorem*{nlem}{Lemma}
\newtheorem*{nprop}{Proposition}
\newtheorem*{ncor}{Corollary}
\newtheorem*{nconj}{Conjecture}
\newtheorem*{nclaim}{Claim}
\theoremstyle{remark}
\newtheorem*{define}{Definition}
\newtheorem*{nrem}{Remarks}
\newtheorem*{notation}{Notation}
\newtheorem*{note}{Note}
\newtheorem*{ex}{Example}
\newtheorem*{imt}{Important}
\newtheorem*{fact}{Fact}
\newcommand{\vs}{\vspace{0.1in}}
\newcommand{\Lim}[1]{\raisebox{0.5ex}{\scalebox{0.8}{$\displaystyle \lim_{#1}\;$}}}


\title{Arc Length}

\author{Stephen Styles}


\begin{document}
\maketitle


Let $y = f(x)$ be a smooth function (twice differentiable) on the interval ${[ a, b ]}$ then the arc length of $f(x)$ on the interval is given by 
\begin{align*}
s = \int_a^b \sqrt{1+{[f'(x)]}^2} \, dx.
\end{align*} 
We can also consider functions $x=g(y)$. Let $g(y)$ be a smooth function on the interval ${[c,d}]$ then the arc length of $g(y)$ is given by
\begin{align*}
s = \int_c^d \sqrt{1+{[g'(y)]}^2} \, dy.
\end{align*}

Note: The arc length for a parametric equation can be calculated by
\begin{align*}
s = \int \sqrt{\bigg(\frac{dx}{dt}\bigg)^2+\bigg(\frac{dy}{dt}\bigg)^2} \, dt.
\end{align*}
Examples:
\begin{enumerate}
\item Find the arc length of $\displaystyle{f(x) = \frac{x^{3/2}}{3}}$ from $0$ to $5$.
\begin{mdframed}[style=TheoremFrame]
\textit{Solution}:
\begin{align*}
f'(x) &= \frac{x^{1/2}}{2}
\end{align*}
\begin{align*}
s &= \int_0^5 \sqrt{1 + \bigg(\frac{\sqrt{x}}{2}\bigg)^2}\, dx\\
&= \int_0^5 \sqrt{1 + \frac{x}{4}}\, dx\\
&= \int_0^5 \frac{1}{2}(4+x)^{1/2} \, dx\\
&= \frac{1}{3} (4+x)^{3/2} \bigg|_0^5 \\
&= \frac{19}{3}
\end{align*}
\end{mdframed}
\newpage
\item Find the arc length of a function whose derivative is given by $\displaystyle{f'(x) = \frac{1}{2}\bigg(x^2 - \frac{1}{x^2}\bigg)}$ from $\displaystyle{\frac{1}{2}}$ to $2$.
\begin{mdframed}[style=TheoremFrame]
\textit{Solution}:

\begin{align*}
s &= \int_{1/2}^2 \sqrt{1+\bigg(\frac{x^2}{2}-\frac{1}{2x^2}\bigg)^2}\, dx\\
&= \int_{1/2}^2 \sqrt{1+\frac{x^4}{4}-\frac{1}{2}+\frac{1}{4x^4}}\, dx\\
&= \int_{1/2}^2 \sqrt{\frac{x^4}{4}+\frac{1}{2}+\frac{1}{4x^4}}\, dx\\
&= \int_{1/2}^2 \sqrt{\bigg(\frac{x^2}{2}+\frac{1}{2x^2}\bigg)^2}\, dx\\
&= \int_{1/2}^2 \frac{x^2}{2}+\frac{1}{2x^2} \, dx\\
&= \frac{x^3}{6}-\frac{1}{2x} \bigg|_{1/2}^2\\
&= \frac{33}{16}
\end{align*}
\end{mdframed}

\item Simplify
\begin{mdframed}[style=TheoremFrame]
\textit{Solution}:\\


\end{mdframed}

\item Simplify
\begin{mdframed}[style=TheoremFrame]
\textit{Solution}:\\


\end{mdframed}
\end{enumerate}

Questions:
\begin{enumerate}
\item Find the arc length of $f(x) = \ln(\cos(x))$ from $0$ to $\displaystyle{\frac{\pi}{4}}$.
\vspace{10cm}
\item Find the arc length of $f(x) = \cosh(x)$ from $0$ to $1$.
\vspace{4cm}
\end{enumerate}


\end{document}