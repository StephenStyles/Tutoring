\documentclass[16pt]{article}

\usepackage{amsmath,amssymb,amsthm,amsfonts,amscd}
\usepackage{showlabels}
\usepackage{color}
\usepackage{hyperref}
\usepackage[numeric]{amsrefs}
\usepackage{graphicx}

\usepackage[framemethod=TikZ]{mdframed}
%\usepackage[framemethod=default]{mdframed}


%\mdfdefinestyle{box}{%
%rightline=true,
%innerleftmargin=10,
%innerrightmargin=10,
%frametitlerule=true,
%frametitlerulecolor=blue,
%frametitlebackgroundcolor=white,
%frametitlerulewidth=2pt}

\mdfdefinestyle{TheoremFrame}{%
    linecolor=blue,
    outerlinewidth=1,
    roundcorner=15,
    innertopmargin= \baselineskip,
    innerbottommargin= \baselineskip,
    innerrightmargin=10,
    innerleftmargin=10,
    backgroundcolor=white}
    
\mdfdefinestyle{ProofFrame}{%
    linecolor=red,
    outerlinewidth=1,
    roundcorner=15,
    innertopmargin= \baselineskip,
    innerbottommargin= \baselineskip,
    innerrightmargin=10,
    innerleftmargin=10,
    backgroundcolor=white}
    
    
\setlength{\oddsidemargin}{0cm}
\setlength{\evensidemargin}{0cm}
\setlength{\marginparwidth}{0in}
\setlength{\marginparsep}{0in}
\setlength{\marginparpush}{0in}
\setlength{\topmargin}{0in}
\setlength{\headheight}{0pt}
\setlength{\headsep}{0pt}
\setlength{\footskip}{.3in}
\setlength{\textheight}{9.2in}
\setlength{\textwidth}{6.0in}
\setlength{\parskip}{0.25pt}
\setlength{\parindent}{0.25in}

    
\newlength\tindent
\setlength{\tindent}{\parindent}
\setlength{\parindent}{0pt}
\renewcommand{\indent}{\hspace*{\tindent}}

\newtheorem*{nthm}{Theorem}
\newtheorem*{nlem}{Lemma}
\newtheorem*{nprop}{Proposition}
\newtheorem*{ncor}{Corollary}
\newtheorem*{nconj}{Conjecture}
\newtheorem*{nclaim}{Claim}
\theoremstyle{remark}
\newtheorem*{define}{Definition}
\newtheorem*{nrem}{Remarks}
\newtheorem*{notation}{Notation}
\newtheorem*{note}{Note}
\newtheorem*{ex}{Example}
\newtheorem*{imt}{Important}
\newtheorem*{fact}{Fact}
\newcommand{\vs}{\vspace{0.1in}}
\newcommand{\Lim}[1]{\raisebox{0.5ex}{\scalebox{0.8}{$\displaystyle \lim_{#1}\;$}}}


\title{Calculating Volumes Using Cylindrical Shells}

\author{Stephen Styles}
\allowdisplaybreaks

\begin{document}
\maketitle

The formula to calculate the volume using cylindrical shells is:
\begin{align*}
V&= \int_a^b 2\pi\, r(x) f(x) \, dx
\end{align*}
Where $r(x)$ is the radius to the line you are rotating about. Using we just consider rotations around the $y$-axis, so in this cause $r(x)=x$. However, if we consider rotations around the line $x=a$, then our radius would be $r(x)=x-a$.\\

This can also be extended to rotations around the $x$-axis or a line $y=a$ using the formula:
\begin{align*}
V&= \int_c^d 2\pi\, r(y) f(y) \, dy
\end{align*}
Where $r(y)=y$ if we are rotation around the $x$-axis or $r(y) = y-a$ if we are rotating around the line $y=a$\\

Examples:
\begin{enumerate}
\item Find the volume of the solid generated by rotating the region bounded by the curves $y=x$ and $y=x^2$, about the $y$-axis.
\begin{mdframed}[style=TheoremFrame]
\textit{Solution:}
\begin{align*}
x&=x^2 \text{ if and only if } x=0,1
\end{align*}
\begin{align*}
V&= \int_0^1 2 \pi x (x-x^2) \, dx\\
&= 2 \pi \int_0^1 x^2 - x^3 \, dx\\
&= 2 \pi \bigg(\frac{x^3}{3} - \frac{x^4}{4}\bigg) \bigg|_0^1\\
&= \frac{\pi}{6}
\end{align*}
\end{mdframed}
\item Find the volume of the solid generated by rotating the region bounded by the curves $y=\sqrt{x}$ and $y=x$, about the line $x=-4$.
\begin{mdframed}[style=TheoremFrame]
\textit{Solution:}
\begin{align*}
x&=\sqrt{x} \text{ if and only if } x=0,1
\end{align*}
\begin{align*}
V&= \int_0^1 2 \pi (x+4) (\sqrt{x}-x) \, dx\\
&= 2 \pi \int_0^1 x^{3/2} - x^2 + 4 \sqrt{x} - 4x \, dx\\
&= 2 \pi \bigg(\frac{2x^{5/2}}{5} - \frac{x^3}{3} + \frac{8x^{3/2}}{3}-2x^2\bigg) \bigg|_0^1\\
&= \frac{22\pi}{15}
\end{align*}
\end{mdframed}
\item Find the volume of the solid generated by rotating the region bounded by the curves\\ $\displaystyle{y=\frac{1}{x\sqrt{x^2+1}}}$, $x=1$, and $x=2$ about the $y$-axis.
\begin{mdframed}[style=TheoremFrame]
\textit{Solution:}
\begin{align*}
V&= \int_1^2 2 \pi x \frac{1}{x\sqrt{x^2+1}} \, dx\\
&= 2\pi \int_0^1 \frac{1}{\sqrt{x^2+1}} \, dx \\
&= 2\pi \text{sinh}^{-1}(x) \bigg|_1^2\\
&= 2\pi \big(\text{sinh}^{-1}(2)-\text{sinh}^{-1}(1)\big)
\end{align*}
\end{mdframed}

\item Determine the volume of the solid obtained by rotating the region bounded by 
$x=(y-1)^2$ and $x=5y-11$ about the line $y=-1$.
\begin{mdframed}[style=TheoremFrame]
\textit{Solution:}
\begin{align*}
5y-11 &= (y-1)^2\\
\Rightarrow 5y-11 &= y^2 - 2y + 1\\
\Rightarrow 0 &=  y^2 - 7y + 12\\
\Rightarrow 0 &= (y-3)(y-4)\\
\Rightarrow y&= 3,4
\end{align*}

Therefore we know these curves will intersect at the points $(1,2)$ and $(0,1)$
\begin{align*}
V &= 2\pi \int_3^4 (y+1) \big((5y-11)-(y-1)^2\big) \, dx\\
&=  2\pi \int_3^4 -y^3 + 6y^2 - 5y -12 \, dx\\
&= 2\pi \bigg( \frac{-y^4}{4} + 2y^3 - \frac{5y^2}{2} - 12y \bigg) \bigg|_3^4\\
&= \frac{3\pi}{2}
\end{align*}
\end{mdframed}
\end{enumerate}
\newpage
Questions:
\begin{enumerate}
\item Calculate the volume of the solid generated by rotating the region bounded by the curves\\ $\displaystyle{y=\frac{1}{x}}$, $x=1$, and $x=3$ about the $y$-axis.
\begin{mdframed}[style=TheoremFrame]
\textit{Solution:}
\begin{align*}
V&= \int_1^3 2\pi x \frac{1}{x} \, dx\\
&= \int_1^3 2\pi \, dx\\
&= 2\pi x \bigg|_1^3\\
&= 4\pi
\end{align*}
\end{mdframed}
\item Calculate the volume of the solid generated by rotating the region bounded by the curves\\ $\displaystyle{y=\bigg(\frac{x}{2}\bigg)^2}$, $y=4$, and $x=0$ about the $x$-axis.
\begin{mdframed}[style=TheoremFrame]
\textit{Solution:}
\begin{align*}
y &= \bigg(\frac{x}{2}\bigg)^2\\
\Rightarrow \sqrt{y} &= \frac{x}{2}\\
\Rightarrow 2 \sqrt{y} &= x
\end{align*}
\begin{align*}
V&= \int_0^4 2\pi y 2 \sqrt{y}\, dy\\
&= 4\pi \int_0^4 y^{3/2} \, dy\\
&= 4\pi \frac{2}{5} y^{5/2} \bigg|_0^4\\
&= \frac{256\pi}{5}
\end{align*}
\end{mdframed}
\newpage
\item Determine the volume of the solid obtained by rotating the region bounded by 
$x=(y-3)^2$ and $x=-4y+9$ about the line $y=-2$.

\begin{mdframed}[style=TheoremFrame]
\textit{Solution:}

\begin{align*}
-4y+9 &= (y-3)^2\\
\Rightarrow -4y+9 &= y^2-6y+9\\
\Rightarrow y^2-y &= 0\\
\Rightarrow y(y-2) &=0\\
\Rightarrow y &= 0,2
\end{align*}
\begin{align*}
V&= 2\pi \int_0^2 (y+2)\bigg((-4y+9) -(y-3)^2\bigg) \, dx\\
&= 2\pi \int_0^2 (y+2)(2y-y^2)\,dy\\
&= 2\pi \int_0^2 2y^2-y^3+4y-2y^2 \, dy\\
&= 2\pi \int_0^2 4y-y^3 \, dy\\
&= 2\pi \bigg(2y^2 - \frac{y^4}{4}\bigg) \bigg|_0^2\\
&= 8\pi
\end{align*}
\end{mdframed}
\item Determine the volume of the solid generated by rotating the region bounded by the curve $y=\sin(x^2)$ from $x=0$ to $x=\sqrt{\pi}$, about the $y$-axis.
\begin{mdframed}[style=TheoremFrame]
\textit{Solution:}
\begin{align*}
V&= \int_0^{\sqrt{\pi}} 2\pi x \sin(x^2) \, dx\\
&= -\pi \cos(x^2) \bigg|_0^{\sqrt{\pi}}\\
&= 2\pi
\end{align*}

\end{mdframed}
\end{enumerate}
\end{document}