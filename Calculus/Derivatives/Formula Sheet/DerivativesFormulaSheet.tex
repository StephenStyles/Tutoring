\documentclass[16pt]{article}

\usepackage{amsmath,amssymb,amsthm,amsfonts,amscd}
\usepackage{showlabels}
\usepackage{color}
\usepackage{hyperref}
\usepackage[numeric]{amsrefs}
\usepackage{graphicx}

\DeclareMathOperator{\sech}{sech}
\DeclareMathOperator{\csch}{csch}

\usepackage[framemethod=TikZ]{mdframed}
%\usepackage[framemethod=default]{mdframed}


%\mdfdefinestyle{box}{%
%rightline=true,
%innerleftmargin=10,
%innerrightmargin=10,
%frametitlerule=true,
%frametitlerulecolor=blue,
%frametitlebackgroundcolor=white,
%frametitlerulewidth=2pt}

\mdfdefinestyle{TheoremFrame}{%
    linecolor=blue,
    outerlinewidth=1,
    roundcorner=15,
    innertopmargin= \baselineskip,
    innerbottommargin= \baselineskip,
    innerrightmargin=10,
    innerleftmargin=10,
    backgroundcolor=white}
    
\mdfdefinestyle{ProofFrame}{%
    linecolor=red,
    outerlinewidth=1,
    roundcorner=15,
    innertopmargin= \baselineskip,
    innerbottommargin= \baselineskip,
    innerrightmargin=10,
    innerleftmargin=10,
    backgroundcolor=white}
    
    
\setlength{\oddsidemargin}{0cm}
\setlength{\evensidemargin}{0cm}
\setlength{\marginparwidth}{0in}
\setlength{\marginparsep}{0in}
\setlength{\marginparpush}{0in}
\setlength{\topmargin}{0in}
\setlength{\headheight}{0pt}
\setlength{\headsep}{0pt}
\setlength{\footskip}{.3in}
\setlength{\textheight}{9.2in}
\setlength{\textwidth}{6.0in}
\setlength{\parskip}{0.25pt}
\setlength{\parindent}{0.25in}

    
\newlength\tindent
\setlength{\tindent}{\parindent}
\setlength{\parindent}{0pt}
\renewcommand{\indent}{\hspace*{\tindent}}

\newtheorem*{nthm}{Theorem}
\newtheorem*{nlem}{Lemma}
\newtheorem*{nprop}{Proposition}
\newtheorem*{ncor}{Corollary}
\newtheorem*{nconj}{Conjecture}
\newtheorem*{nclaim}{Claim}
\theoremstyle{remark}
\newtheorem*{define}{Definition}
\newtheorem*{nrem}{Remarks}
\newtheorem*{notation}{Notation}
\newtheorem*{note}{Note}
\newtheorem*{ex}{Example}
\newtheorem*{imt}{Important}
\newtheorem*{fact}{Fact}
\newcommand{\vs}{\vspace{0.1in}}
\newcommand{\Lim}[1]{\raisebox{0.5ex}{\scalebox{0.8}{$\displaystyle \lim_{#1}\;$}}}


\title{Derivatives Formula Sheet}

\author{Stephen Styles}


\begin{document}

\subsection*{Definition of a Derivative}

\begin{align*}
\frac{d}{dx} f(x) &= \lim_{x\rightarrow a} \frac{f(x)-f(a)}{x-a}
\end{align*}

\subsection*{Basic Rules}

Scalar Multiplication Rule: $\displaystyle{\big{[ c f(x) \big]}' = c f'(x)}$\\

Sum Rule: $\displaystyle{\big{[f(x)+g(x)\big]}' = f'(x) + g'(x) }$\\

Chain Rule: $\displaystyle{ \big{[ f(g(x)) \big]}' = f'\big(g(x)\big)\cdot g'(x) }$\\

Product Rule: $\displaystyle{ \big{[f(x)\cdot g(x)\big]}' = f'(x)\cdot g(x) + f(x) \cdot g'(x) }$\\

Quotient Rule: $\displaystyle{ \bigg{[ \displaystyle{\frac{f(x)}{g(x)}} \bigg]}' = \frac{g(x)\cdot f'(x) - f(x)\cdot g'(x)}{g(x)^2} }$\\

Inverse Rule: $\displaystyle{ \big{[ f^{-1} (x) \big]}' = \frac{1}{f'\big( f^{-1}(x)\big)}}$

\subsection*{Polynomial Derivatives}

\begin{center}
\begin{tabular}{l l}
$\displaystyle{\frac{d}{dx} C = 0}$, where $C$ is a constant & $\displaystyle{\frac{d}{dx} x^n = nx^{n-1}}$
\end{tabular}
\end{center}

\subsection*{Exponential Derivatives}

\begin{center}
\begin{tabular}{l l}
$\displaystyle{\frac{d}{dx} e^x = e^x}$ & $\displaystyle{\frac{d}{dx} a^x = a^x \ln(a)}$\\[3ex]
$\displaystyle{\frac{d}{dx} \ln(x) = \frac{1}{x}}$ & $\displaystyle{\frac{d}{dx} \log_a(x) = \frac{1}{x \ln(a)}}$
\end{tabular}
\end{center} 
\subsection*{Trig Derivatives}

\begin{center}
\begin{tabular}{l l}
$\displaystyle{\frac{d}{dx} \sin\big(f(x)\big) = \cos\big(f(x)\big)f'(x)}$ & $\displaystyle{\frac{d}{dx} \cos\big(f(x)\big) = -\sin\big(f(x)\big)f'(x)}$\\[2ex]
$\displaystyle{\frac{d}{dx} \sec\big(f(x)\big) = \sec\big(f(x)\big)\tan\big(f(x)\big)f'(x)}$ & $\displaystyle{\frac{d}{dx} \csc\big(f(x)\big) = -\csc\big(f(x)\big)\cot\big(f(x)\big)f'(x)}$\\[2ex]
$\displaystyle{\frac{d}{dx} \tan\big(f(x)\big) = \sec^2\big(f(x)\big)f'(x)}$ & $\displaystyle{\frac{d}{dx} \cot\big(f(x)\big) = -\csc^2\big(f(x)\big)f'(x)}$
\end{tabular}
\end{center}

\subsection*{Inverse Trig Derivatives}

\begin{center}
\begin{tabular}{l l}
$\displaystyle{\frac{d}{dx} \sin^{-1}\big(f(x)\big) = \frac{f'(x)}{\sqrt{1-\big(f(x)\big)^2}}}$ & $\displaystyle{\frac{d}{dx} \cos^{-1}\big(f(x)\big) = \frac{-f'(x)}{\sqrt{1-\big(f(x)\big)^2}}}$\\[5ex]
$\displaystyle{\frac{d}{dx} \sec^{-1}\big(f(x)\big) = \frac{f'(x)}{\big|f(x)\big|\sqrt{\big(f(x)\big)^2-1}}}$ & $\displaystyle{\frac{d}{dx} \csc^{-1}\big(f(x)\big) = \frac{-f'(x)}{\big|f(x)\big|\sqrt{\big(f(x)\big)^2-1}}}$\\[5ex]
$\displaystyle{\frac{d}{dx} \tan^{-1}\big(f(x)\big) = \frac{f'(x)}{1+\big(f(x)\big)^2}}$ & $\displaystyle{\frac{d}{dx} \cot^{-1}\big(f(x)\big) = \frac{-f'(x)}{1+\big(f(x)\big)^2}}$
\end{tabular}
\end{center}
\newpage
\subsection*{Hyperbolic Trig Derivatives}
\begin{center}
\begin{tabular}{l l}
$\displaystyle{\frac{d}{dx} \sinh\big(f(x)\big) = \cosh\big(f(x)\big)f'(x)}$ & $\displaystyle{\frac{d}{dx} \cosh\big(f(x)\big) = \sinh\big(f(x)\big)f'(x)}$\\[2ex]
$\displaystyle{\frac{d}{dx} \sech\big(f(x)\big) = -\sech\big(f(x)\big)\tanh\big(f(x)\big)f'(x)}$ & $\displaystyle{\frac{d}{dx} \csch\big(f(x)\big) = -\csch\big(f(x)\big)\coth\big(f(x)\big)f'(x)}$\\[2ex]
$\displaystyle{\frac{d}{dx} \tanh\big(f(x)\big) = \sech^2\big(f(x)\big)f'(x)}$ & $\displaystyle{\frac{d}{dx} \coth\big(f(x)\big) = -\csch^2\big(f(x)\big)f'(x)}$
\end{tabular}
\end{center}
\subsection*{Inverse Hyperbolic Trig Derivatives}

\begin{center}
\begin{tabular}{l l}
$\displaystyle{\frac{d}{dx} \sinh^{-1}\big(f(x)\big) = \frac{f'(x)}{\sqrt{1+\big(f(x)\big)^2}}}$ & $\displaystyle{\frac{d}{dx} \cosh^{-1}\big(f(x)\big) = \frac{f'(x)}{\sqrt{\big(f(x)\big)^2-1}}}$\\[5ex]
$\displaystyle{\frac{d}{dx} \sech^{-1}\big(f(x)\big) = \frac{-f'(x)}{f(x)\sqrt{1-\big(f(x)\big)^2}}}$ & $\displaystyle{\frac{d}{dx} \csch^{-1}\big(f(x)\big) = \frac{-f'(x)}{\big|f(x)\big|\sqrt{\big(f(x)\big)^2+1}}}$\\[5ex]
$\displaystyle{\frac{d}{dx} \tanh^{-1}\big(f(x)\big) = \frac{f'(x)}{1-\big(f(x)\big)^2}}$ & $\displaystyle{\frac{d}{dx} \coth^{-1}\big(f(x)\big) = \frac{f'(x)}{1-\big(f(x)\big)^2}}$
\end{tabular}
\end{center}
\end{document}