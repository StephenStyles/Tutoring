\documentclass[16pt]{article}

\usepackage{amsmath,amssymb,amsthm,amsfonts,amscd}
\usepackage{showlabels}
\usepackage{color}
\usepackage{hyperref}
\usepackage[numeric]{amsrefs}
\usepackage{graphicx}

\usepackage[framemethod=TikZ]{mdframed}
%\usepackage[framemethod=default]{mdframed}


%\mdfdefinestyle{box}{%
%rightline=true,
%innerleftmargin=10,
%innerrightmargin=10,
%frametitlerule=true,
%frametitlerulecolor=blue,
%frametitlebackgroundcolor=white,
%frametitlerulewidth=2pt}

\mdfdefinestyle{TheoremFrame}{%
    linecolor=blue,
    outerlinewidth=1,
    roundcorner=15,
    innertopmargin= \baselineskip,
    innerbottommargin= \baselineskip,
    innerrightmargin=10,
    innerleftmargin=10,
    backgroundcolor=white}
    
\mdfdefinestyle{ProofFrame}{%
    linecolor=red,
    outerlinewidth=1,
    roundcorner=15,
    innertopmargin= \baselineskip,
    innerbottommargin= \baselineskip,
    innerrightmargin=10,
    innerleftmargin=10,
    backgroundcolor=white}
    
    
\setlength{\oddsidemargin}{0cm}
\setlength{\evensidemargin}{0cm}
\setlength{\marginparwidth}{0in}
\setlength{\marginparsep}{0in}
\setlength{\marginparpush}{0in}
\setlength{\topmargin}{0in}
\setlength{\headheight}{0pt}
\setlength{\headsep}{0pt}
\setlength{\footskip}{.3in}
\setlength{\textheight}{9.2in}
\setlength{\textwidth}{6.0in}
\setlength{\parskip}{0.25pt}
\setlength{\parindent}{0.25in}

    
\newlength\tindent
\setlength{\tindent}{\parindent}
\setlength{\parindent}{0pt}
\renewcommand{\indent}{\hspace*{\tindent}}

\newtheorem*{nthm}{Theorem}
\newtheorem*{nlem}{Lemma}
\newtheorem*{nprop}{Proposition}
\newtheorem*{ncor}{Corollary}
\newtheorem*{nconj}{Conjecture}
\newtheorem*{nclaim}{Claim}
\theoremstyle{remark}
\newtheorem*{define}{Definition}
\newtheorem*{nrem}{Remarks}
\newtheorem*{notation}{Notation}
\newtheorem*{note}{Note}
\newtheorem*{ex}{Example}
\newtheorem*{imt}{Important}
\newtheorem*{fact}{Fact}
\newcommand{\vs}{\vspace{0.1in}}
\newcommand{\Lim}[1]{\raisebox{0.5ex}{\scalebox{0.8}{$\displaystyle \lim_{#1}\;$}}}


\title{BEDMAS}

\author{Stephen Styles}


\begin{document}
\maketitle
BEDMAS is an acronym that reminds us of the correct order of operations:\\
\begin{center}
\begin{tabular}{ l l }
 \textcolor{red}{B}rackets & \textcolor{red}{First Priority}  \\ 
 \textcolor{green}{E}xponents & \textcolor{green}{Second Priority } \\  
 \textcolor{blue}{D}ivision & \textcolor{blue}{Third Priority}  \\
 \textcolor{blue}{M}ultiplication & \textcolor{blue}{Third Priority}\\
 \textcolor{purple}{A}ddition & \textcolor{purple}{Fourth Priority}\\
 \textcolor{purple}{S}ubtraction & \textcolor{purple}{Fourth Priority}\\    
\end{tabular}
\end{center}
This acronym tells us which order to evaluate our expressions. When an expression has multiple operations that are of the same priority, we solve them from left to right.\\

Examples:
\begin{enumerate}
\item Simplify $(5\times 6)\times1^2 \div3-8+4$
\begin{mdframed}[style=TheoremFrame]
\textit{Solution:}\\

\begin{align*}
(5\times 6)\times1^2 \div3-8+4 &= 30 \times1^2 \div3-8+4 & \text{Simplifying the bracket}\\
&= 30 \times1 \div3-8+4 & \text{Solving the exponent}\\
&= 30 \div 3 -8+4 &\text{Multiplying first since its on the left}\\
&= 10 - 8 + 4 & \text{Dividing because its the highest priority}\\
&= 2+4 & \text{Subtracting since it comes first from the left}\\
&= 6 &
\end{align*}
\end{mdframed}
\item Simplify $(6\times 2 - 5) \times \big((3+9)\times 4 \div 8 \big) \div 1^2-7^2$
\begin{mdframed}[style=TheoremFrame]
\textit{Solution:}\\

\end{mdframed}
\item Simplify $\displaystyle{7^3 + 3\times (4 \times 5 - 2^2 -28 \div 7 +3) + 9 + \frac{3+5}{3-1}}$
\begin{mdframed}[style=TheoremFrame]
\textit{Solution:}\\

\end{mdframed}
\item Simplify $\displaystyle{(6\div (2+4)^{(2^2-4)})^2 + \big(6\times(2+1-2)\big)^2 \div 3}$
\begin{mdframed}[style=TheoremFrame]
\textit{Solution:}\\

\end{mdframed}
\end{enumerate}
\end{document}