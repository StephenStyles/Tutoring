\documentclass[16pt]{article}

\usepackage{amsmath,amssymb,amsthm,amsfonts,amscd}
\usepackage{showlabels}
\usepackage{color}
\usepackage{hyperref}
\usepackage[numeric]{amsrefs}
\usepackage{graphicx}

\usepackage[framemethod=TikZ]{mdframed}
%\usepackage[framemethod=default]{mdframed}


%\mdfdefinestyle{box}{%
%rightline=true,
%innerleftmargin=10,
%innerrightmargin=10,
%frametitlerule=true,
%frametitlerulecolor=blue,
%frametitlebackgroundcolor=white,
%frametitlerulewidth=2pt}

\mdfdefinestyle{TheoremFrame}{%
    linecolor=blue,
    outerlinewidth=1,
    roundcorner=15,
    innertopmargin= \baselineskip,
    innerbottommargin= \baselineskip,
    innerrightmargin=10,
    innerleftmargin=10,
    backgroundcolor=white}
    
\mdfdefinestyle{ProofFrame}{%
    linecolor=red,
    outerlinewidth=1,
    roundcorner=15,
    innertopmargin= \baselineskip,
    innerbottommargin= \baselineskip,
    innerrightmargin=10,
    innerleftmargin=10,
    backgroundcolor=white}
    
    
\setlength{\oddsidemargin}{0cm}
\setlength{\evensidemargin}{0cm}
\setlength{\marginparwidth}{0in}
\setlength{\marginparsep}{0in}
\setlength{\marginparpush}{0in}
\setlength{\topmargin}{0in}
\setlength{\headheight}{0pt}
\setlength{\headsep}{0pt}
\setlength{\footskip}{.3in}
\setlength{\textheight}{9.2in}
\setlength{\textwidth}{6.0in}
\setlength{\parskip}{0.25pt}
\setlength{\parindent}{0.25in}
\allowdisplaybreaks
    
\newlength\tindent
\setlength{\tindent}{\parindent}
\setlength{\parindent}{0pt}
\renewcommand{\indent}{\hspace*{\tindent}}

\newtheorem*{nthm}{Theorem}
\newtheorem*{nlem}{Lemma}
\newtheorem*{nprop}{Proposition}
\newtheorem*{ncor}{Corollary}
\newtheorem*{nconj}{Conjecture}
\newtheorem*{nclaim}{Claim}
\theoremstyle{remark}
\newtheorem*{define}{Definition}
\newtheorem*{nrem}{Remarks}
\newtheorem*{notation}{Notation}
\newtheorem*{note}{Note}
\newtheorem*{ex}{Example}
\newtheorem*{imt}{Important}
\newtheorem*{fact}{Fact}
\newcommand{\vs}{\vspace{0.1in}}
\newcommand{\Lim}[1]{\raisebox{0.5ex}{\scalebox{0.8}{$\displaystyle \lim_{#1}\;$}}}


\title{Mathematical Laws}

\author{Stephen Styles}


\begin{document}

\maketitle

The \textbf{Commutative Law} allows you to switch the order of the terms you are performing an operation on.\\

For addition:
\begin{align*}
a+b &= b +a
\end{align*}
For multiplication:
\begin{align*}
a\times b &= b\times a
\end{align*}

The \textbf{Associative Law} allows you tells us that how we group things doesn't matter.\\

For addition:
\begin{align*}
(a + b) + c &= a + (b + c)
\end{align*}
For multiplication:
\begin{align*}
(a\times b) \times c &= a \times (b \times c)
\end{align*}

The \textbf{Distributive Law} allows you to multiply a number into a larger term, or pull out a common factor form a few terms.

\begin{align*}
a\times (b+c) &= a\times b + a \times c
\end{align*}

Questions:
\begin{enumerate}
\item Simplify $\displaystyle{x^2-7x+2 - 3x-2x^2+9}$
\begin{mdframed}[style=TheoremFrame]
\textit{Solution:}
\begin{align*}
&x^2-7x+2 - 3x-2x^2+9\\
&= x^2 - 2x^2 - 7x - 3x + 2 + 9\\
&= (1-2)x^2 + (-7-3)x + (2+9)\\
&= -x^2-10x+11
\end{align*}
\end{mdframed}
\newpage

\item Simplify $\displaystyle{-7x^4+5x^3+13x-1 - 5x^5 +7x^4 - 10x^3 -7x-3}$
\begin{mdframed}[style=TheoremFrame]
\textit{Solution:}
\begin{align*}
&-7x^4+5x^3+13x-1 - 5x^5 +7x^4 - 10x^3 -7x-3\\
&= -5x^5 - 7x^4+7x^4+5x^3-10x^3+13x-7x-1-3\\
&= -5x^5 +(-7+7)x^4 + (5-10)x^3+(13-7)x+(-1-3)\\
&= -5x^5 + -5x^3+6x-4
\end{align*}
\end{mdframed}

\item Simplify $\displaystyle{x^3+2x^2-4x^3+x-13x^2+3-2x+4}$
\begin{mdframed}[style=TheoremFrame]
\textit{Solution:}
\begin{align*}
&x^3+2x^2-4x^3+x-13x^2+3-2x+4\\
&= x^3-4x^3+2x^2-13x^2+x-2x+3+4\\
&= (1-4)x^3+(2-13)x^2+(1-2)x+(3+4)\\
&= -3x^3-11x^2-x+7
\end{align*}
\end{mdframed}

\item Simplify $\displaystyle{4\times(5x^2+6x-9)}$
\begin{mdframed}[style=TheoremFrame]
\textit{Solution:}
\begin{align*}
&4\times(5x^2+6x-9)\\
&=(4\times 5x^2)+(4\times 6x) - (4\times 9)\\
&= 20x^2 + 24x - 36
\end{align*}
\end{mdframed}
\item Simplify $\displaystyle{-2\times(x^2-7x-17)}$
\begin{mdframed}[style=TheoremFrame]
\textit{Solution:}
\begin{align*}
&-2\times(x^2-7x-17)\\
&= (-2\times x^2) + (-2\times -7x) + (-2 \times -17)\\
&= -2x^2 + 14x + 34
\end{align*}
\end{mdframed}
\item Simplify $\displaystyle{3\times(4x^3-2x^2-5x+12)}$
\begin{mdframed}[style=TheoremFrame]
\textit{Solution:}
\begin{align*}
&3\times(4x^3-2x^2-5x+12)\\
&= (3\times 4x^3) + (3\times -2x^2) + (3 \times -5x) + (3\times 12)\\
&= 12x^3 -6x^2 -15x+36
\end{align*}
\end{mdframed}
\newpage
\item Simplify $\displaystyle{2\times(2x^2-5x+6)-3\times(x^2+3x+4)}$
\begin{mdframed}[style=TheoremFrame]
\textit{Solution:}
\begin{align*}
&2\times(2x^2-5x+6)-3\times(x^2+3x+4)\\
&= (2\times 2x^2) + (2\times -5x) + (2\times 6) + (-3\times x^2) + (-3 \times 3x) + (-3\times 4)\\
&= 4x^2 -10x + 12 -3x^2 - 9x - 12\\
&= 4x^2 - 3x^2 -10x - 9x + 12 - 12\\
&= (4-3)x^2 + (-10-9)x + (12-12)\\
&= x^2 - 19x
\end{align*}
\end{mdframed}
\item Simplify $\displaystyle{-1\times(7x^2-6x-6+4x^2+2x-5)}$
\begin{mdframed}[style=TheoremFrame]
\textit{Solution:}
\begin{align*}
&-1\times(7x^2-6x-6+4x^2+2x-5)\\
&= -1 \times (7x^2 + 4x^2 -6x + 2x -6 - 5)\\
&= -1 \times \big( (7+4)x^2 + (-6+2)x + (-6-5)\big)\\
&= -1 \times ( 11x^2 -4x - 11 )\\
&= (-1 \times 11x^2) + (-1 \times -4x) + (-1\times -11)\\
&= -11x^2 + 4x + 11
\end{align*}
\end{mdframed}
\item Simplify $\displaystyle{4a-3b + 2\times\bigg(\frac{3}{2}a -\frac{3}{4}b\bigg)}$
\begin{mdframed}[style=TheoremFrame]
\textit{Solution:}
\begin{align*}
&4a-3b + 2\times\bigg(\frac{3}{2}a -\frac{3}{4}b\bigg)\\
&= 4a-3b + 2\times\bigg(\frac{3}{2}a\bigg) - 2\times \bigg(\frac{3}{4}b\bigg)\\
&= 4a - 3b + 3a - \frac{3}{2}b\\
&= 4a+3a - 3b - \frac{3}{2}b\\
&= (4+3)a + \bigg(-3-\frac{3}{2}\bigg)b\\
&= 7a + \bigg(-\frac{6}{2}-\frac{3}{2}\bigg)b\\
&= 7a -\frac{9}{2}b
\end{align*}
\end{mdframed}
\newpage
\item Simplify $\displaystyle{\frac{5x^2-6x+9-x^2-10x+5}{8}}$
\begin{mdframed}[style=TheoremFrame]
\textit{Solution:}
\begin{align*}
&\frac{5x^2-6x+9-x^2-10x+5}{8}\\[2ex]
&= \frac{5x^2-x^2-6x-10x+9+5}{8}\\[2ex]
&= \frac{(5-1)x^2+(-6-10)x+(9+5)}{8}\\[2ex]
&= \frac{4x^2-16x+14}{8}\\[2ex]
&= \frac{4}{8}x^2-\frac{16}{8}x + \frac{14}{8}\\[2ex]
&= \frac{1}{2}x^2 -2x + \frac{7}{4}
\end{align*}
\end{mdframed}
\end{enumerate}
\end{document}