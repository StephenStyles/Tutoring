\documentclass[16pt]{article}

\usepackage{amsmath,amssymb,amsthm,amsfonts,amscd}
\usepackage{showlabels}
\usepackage{color}
\usepackage{hyperref}
\usepackage[numeric]{amsrefs}
\usepackage{graphicx}

\usepackage[framemethod=TikZ]{mdframed}
%\usepackage[framemethod=default]{mdframed}


%\mdfdefinestyle{box}{%
%rightline=true,
%innerleftmargin=10,
%innerrightmargin=10,
%frametitlerule=true,
%frametitlerulecolor=blue,
%frametitlebackgroundcolor=white,
%frametitlerulewidth=2pt}

\mdfdefinestyle{TheoremFrame}{%
    linecolor=blue,
    outerlinewidth=1,
    roundcorner=15,
    innertopmargin= \baselineskip,
    innerbottommargin= \baselineskip,
    innerrightmargin=10,
    innerleftmargin=10,
    backgroundcolor=white}
    
\mdfdefinestyle{ProofFrame}{%
    linecolor=red,
    outerlinewidth=1,
    roundcorner=15,
    innertopmargin= \baselineskip,
    innerbottommargin= \baselineskip,
    innerrightmargin=10,
    innerleftmargin=10,
    backgroundcolor=white}
    
    
\setlength{\oddsidemargin}{0cm}
\setlength{\evensidemargin}{0cm}
\setlength{\marginparwidth}{0in}
\setlength{\marginparsep}{0in}
\setlength{\marginparpush}{0in}
\setlength{\topmargin}{0in}
\setlength{\headheight}{0pt}
\setlength{\headsep}{0pt}
\setlength{\footskip}{.3in}
\setlength{\textheight}{9.2in}
\setlength{\textwidth}{6.0in}
\setlength{\parskip}{0.25pt}
\setlength{\parindent}{0.25in}
\allowdisplaybreaks
    
\newlength\tindent
\setlength{\tindent}{\parindent}
\setlength{\parindent}{0pt}
\renewcommand{\indent}{\hspace*{\tindent}}

\newtheorem*{nthm}{Theorem}
\newtheorem*{nlem}{Lemma}
\newtheorem*{nprop}{Proposition}
\newtheorem*{ncor}{Corollary}
\newtheorem*{nconj}{Conjecture}
\newtheorem*{nclaim}{Claim}
\theoremstyle{remark}
\newtheorem*{define}{Definition}
\newtheorem*{nrem}{Remarks}
\newtheorem*{notation}{Notation}
\newtheorem*{note}{Note}
\newtheorem*{ex}{Example}
\newtheorem*{imt}{Important}
\newtheorem*{fact}{Fact}
\newcommand{\vs}{\vspace{0.1in}}
\newcommand{\Lim}[1]{\raisebox{0.5ex}{\scalebox{0.8}{$\displaystyle \lim_{#1}\;$}}}


\title{Mathematical Laws}

\author{Stephen Styles}


\begin{document}

\maketitle

The \textbf{Commutative Law} allows you to switch the order of the terms you are performing an operation on.\\

For addition:
\begin{align*}
a+b &= b +a
\end{align*}
For multiplication:
\begin{align*}
a\times b &= b\times a
\end{align*}

The \textbf{Associative Law} allows you tells us that how we group things doesn't matter.\\

For addition:
\begin{align*}
(a + b) + c &= a + (b + c)
\end{align*}
For multiplication:
\begin{align*}
(a\times b) \times c &= a \times (b \times c)
\end{align*}

The \textbf{Distributive Law} allows you to multiply a number into a larger term, or pull out a common factor form a few terms.

\begin{align*}
a\times (b+c) &= a\times b + a \times c
\end{align*}

Questions:
\begin{enumerate}
\item Simplify $\displaystyle{x^2-7x+2 - 3x-2x^2+9}$
\vspace{3cm}

\item Simplify $\displaystyle{-7x^4+5x^3+13x-1 - 5x^5 +7x^4 - 10x^3 -7x-3}$
\newpage

\item Simplify $\displaystyle{x^3+2x^2-4x^3+x-13x^2+3-2x+4}$
\vspace{5cm}

\item Simplify $\displaystyle{4\times(5x^2+6x-9)}$
\vspace{5cm}
\item Simplify $\displaystyle{-2\times(x^2-7x-17)}$
\vspace{5cm}
\item Simplify $\displaystyle{3\times(4x^3-2x^2-5x+12)}$
\vspace{5cm}
\newpage
\item Simplify $\displaystyle{2\times(2x^2-5x+6)-3\times(x^2+3x+4)}$
\vspace{5cm}
\item Simplify $\displaystyle{-1\times(7x^2-6x-6+4x^2+2x-5)}$
\vspace{5cm}
\item Simplify $\displaystyle{4a-3b + 2\times\bigg(\frac{3}{2}a -\frac{3}{4}b\bigg)}$
\vspace{5cm}
\item Simplify $\displaystyle{\frac{5x^2-6x+9-x^2-10x+5}{8}}$
\vspace{5cm}
\end{enumerate}
\end{document}