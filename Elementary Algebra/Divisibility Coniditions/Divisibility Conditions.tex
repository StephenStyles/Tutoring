\documentclass[16pt]{article}

\usepackage{amsmath,amssymb,amsthm,amsfonts,amscd}
\usepackage{showlabels}
\usepackage{color}
\usepackage{hyperref}
\usepackage[numeric]{amsrefs}
\usepackage{graphicx}

\usepackage[framemethod=TikZ]{mdframed}
%\usepackage[framemethod=default]{mdframed}


%\mdfdefinestyle{box}{%
%rightline=true,
%innerleftmargin=10,
%innerrightmargin=10,
%frametitlerule=true,
%frametitlerulecolor=blue,
%frametitlebackgroundcolor=white,
%frametitlerulewidth=2pt}

\mdfdefinestyle{TheoremFrame}{%
    linecolor=blue,
    outerlinewidth=1,
    roundcorner=15,
    innertopmargin= \baselineskip,
    innerbottommargin= \baselineskip,
    innerrightmargin=10,
    innerleftmargin=10,
    backgroundcolor=white}
    
\mdfdefinestyle{ProofFrame}{%
    linecolor=red,
    outerlinewidth=1,
    roundcorner=15,
    innertopmargin= \baselineskip,
    innerbottommargin= \baselineskip,
    innerrightmargin=10,
    innerleftmargin=10,
    backgroundcolor=white}
    
    
\setlength{\oddsidemargin}{0cm}
\setlength{\evensidemargin}{0cm}
\setlength{\marginparwidth}{0in}
\setlength{\marginparsep}{0in}
\setlength{\marginparpush}{0in}
\setlength{\topmargin}{0in}
\setlength{\headheight}{0pt}
\setlength{\headsep}{0pt}
\setlength{\footskip}{.3in}
\setlength{\textheight}{9.2in}
\setlength{\textwidth}{6.0in}
\setlength{\parskip}{0.25pt}
\setlength{\parindent}{0.25in}

    
\newlength\tindent
\setlength{\tindent}{\parindent}
\setlength{\parindent}{0pt}
\renewcommand{\indent}{\hspace*{\tindent}}

\newtheorem*{nthm}{Theorem}
\newtheorem*{nlem}{Lemma}
\newtheorem*{nprop}{Proposition}
\newtheorem*{ncor}{Corollary}
\newtheorem*{nconj}{Conjecture}
\newtheorem*{nclaim}{Claim}
\theoremstyle{remark}
\newtheorem*{define}{Definition}
\newtheorem*{nrem}{Remarks}
\newtheorem*{notation}{Notation}
\newtheorem*{note}{Note}
\newtheorem*{ex}{Example}
\newtheorem*{imt}{Important}
\newtheorem*{fact}{Fact}
\newcommand{\vs}{\vspace{0.1in}}
\newcommand{\Lim}[1]{\raisebox{0.5ex}{\scalebox{0.8}{$\displaystyle \lim_{#1}\;$}}}


\title{Divisibility Rules}

\author{Stephen Styles}


\begin{document}
\maketitle

For all the common divisors, there is a rule to tell if an integer is divisible by the divisor of interest. From $1$ to $10$ these rules are:

\begin{table}[h!]
\centering
\begin{tabular}{||c | c||} 
 \hline
 Divisor & Rule \\ [0.5ex] 
 \hline\hline
 $1$ & All numbers are divisible by $1$.\\
 \hline
 $2$ & The last digit is either $\{0,2,4,6,8\}$, i.e. the integer is even.\\
 \hline
 $3$ & The sum of the digits must be divisible by $3$.\\ 
 \hline
 $4$ & The last two digits form a number that is divisible by 4.\\
 & \textbf{OR} The integer is divisible by $2$ twice \\
 \hline
 $5$ & The last digit is $\{0, 5\}$.\\
 \hline
 $6$ & It is divisible by $2$ and by $3$. \\
 \hline
 $7$ & Adding the last two digits to twice the rest gives a multiple of $7$.\\
 & \textbf{OR} Adding the last digit to $3$ times the rest gives a multiple of $7$. \\
 \hline
 $8$ & The last three digits are divisible by $8.$\\
 & \textbf{OR} If the hundreds digit is even, the number formed\\
 & by the last two digits must be divisible by $8$.\\
 & \textbf{OR} If the hundreds digit is odd, the number obtained\\
 & by the last two digits plus 4 must be divisible by $8$.\\
 & \textbf{OR} Add the last digit to twice the rest.\\
 & The result must be divisible by 8.\\
 \hline
 $9$ & The sum of the digits must be divisible by $9$. \\
 \hline
 $10$ & The last digit is $0$.\\
 \hline
\end{tabular}
\caption{Rules for each divisor}
\label{table:1}
\end{table}

To test if an integer is divisible by the divisor of interest, just check the rule to see if the condition is met. If the condition is not met, then the integer is not divisible by the divisor.\\

Examples:
\begin{enumerate}
\item Test if $943,314$ is divisible by $2,3,4,$ or $5$.
\begin{mdframed}[style = TheoremFrame]
\textit{Solution}:\\

2: The last digit is $4$, therefore $943,314$ is even and thus divisible by $2$.\\

3: $9+4+3+3+1+4 = 24$, $24 = 3 \cdot 8$, therefore $943,314$ is divisible by $3$.\\

4: $14$ is not divisible by $4$, therefore $943,314$ is not divisible by $4$.\\

5: The last digit is not $0$ or $5$, therefore $943,314$ is not divisible by $5$.
\end{mdframed}
\item Test if $15,148$ is divisible by $7$.
\begin{mdframed}[style = TheoremFrame]
\textit{Solution}:\\

$48+(2 \cdot 151) = 48 + 302 = 350$, from here we can repeat the process.\\
$50 + (2 \cdot 3) = 56$ where $56$ is divisible by $7$, therefore $15,148$ is divisible by $7$.
\end{mdframed}
\item Show that $3,579$ is divisible by $3$ but not by $6$ or $9$.
\begin{mdframed}[style = TheoremFrame]
\textit{Solution}:\\

3: $3+5+7+9=24$ which is divisible by $3$, therefore $3,579$ is divisible by $3$.\\

6: The last digit is not $\{0,2,4,6,8\}$, therefore $3,579$ is not divisible by $2$ which means it can not be divisible by $6$.\\

9: $3+5+7+9 = 24$ which is not divisible by $9$, therefore $3,579$ is divisible by $9$.
\end{mdframed}
\end{enumerate}
Problem Set:
\begin{enumerate}
\item Show that $761,982,908$ is divisible by $4$.
\vspace{3cm}
\item Show that $213$ is not divisible by $7$ but is divisible by $3$.
\vspace{3cm}
\item Show that $16,722$ is divisible by $2$ and $9$.
\vspace{3cm}
\newpage
\item Show that $15,275$ is divisible by $5$ but not by $3$.
\vspace{3cm}
\item Show that $63,204$ is divisible by $6$. 
\end{enumerate}
\end{document}