\documentclass[16pt]{article}

\usepackage{amsmath,amssymb,amsthm,amsfonts,amscd}
\usepackage{showlabels}
\usepackage{color}
\usepackage{hyperref}
\usepackage[numeric]{amsrefs}
\usepackage{graphicx}

\usepackage[framemethod=TikZ]{mdframed}
%\usepackage[framemethod=default]{mdframed}


%\mdfdefinestyle{box}{%
%rightline=true,
%innerleftmargin=10,
%innerrightmargin=10,
%frametitlerule=true,
%frametitlerulecolor=blue,
%frametitlebackgroundcolor=white,
%frametitlerulewidth=2pt}

\mdfdefinestyle{TheoremFrame}{%
    linecolor=blue,
    outerlinewidth=1,
    roundcorner=15,
    innertopmargin= \baselineskip,
    innerbottommargin= \baselineskip,
    innerrightmargin=10,
    innerleftmargin=10,
    backgroundcolor=white}
    
\mdfdefinestyle{ProofFrame}{%
    linecolor=red,
    outerlinewidth=1,
    roundcorner=15,
    innertopmargin= \baselineskip,
    innerbottommargin= \baselineskip,
    innerrightmargin=10,
    innerleftmargin=10,
    backgroundcolor=white}
    
    
\setlength{\oddsidemargin}{0cm}
\setlength{\evensidemargin}{0cm}
\setlength{\marginparwidth}{0in}
\setlength{\marginparsep}{0in}
\setlength{\marginparpush}{0in}
\setlength{\topmargin}{0in}
\setlength{\headheight}{0pt}
\setlength{\headsep}{0pt}
\setlength{\footskip}{.3in}
\setlength{\textheight}{9.2in}
\setlength{\textwidth}{6.0in}
\setlength{\parskip}{0.25pt}
\setlength{\parindent}{0.25in}
\allowdisplaybreaks
    
\newlength\tindent
\setlength{\tindent}{\parindent}
\setlength{\parindent}{0pt}
\renewcommand{\indent}{\hspace*{\tindent}}

\newtheorem*{nthm}{Theorem}
\newtheorem*{nlem}{Lemma}
\newtheorem*{nprop}{Proposition}
\newtheorem*{ncor}{Corollary}
\newtheorem*{nconj}{Conjecture}
\newtheorem*{nclaim}{Claim}
\theoremstyle{remark}
\newtheorem*{define}{Definition}
\newtheorem*{nrem}{Remarks}
\newtheorem*{notation}{Notation}
\newtheorem*{note}{Note}
\newtheorem*{ex}{Example}
\newtheorem*{imt}{Important}
\newtheorem*{fact}{Fact}
\newcommand{\vs}{\vspace{0.1in}}
\newcommand{\Lim}[1]{\raisebox{0.5ex}{\scalebox{0.8}{$\displaystyle \lim_{#1}\;$}}}


\title{Euclidean Algorithm}

\author{Stephen Styles}


\begin{document}

\maketitle

The Euclidean Algorithm is a tool used to determine the greatest common divisor (GCD) of two numbers. It follows this three step procedure:
\begin{enumerate}
\item Let $a$ be the larger of the two numbers in magnitude and $b$ be the smaller.
\item Divide $a$ by $b$ and get the remainder $R$. If $R=0$ then the GCD of $a$ and $b$ is $R$.
\item If $R\not=0$, replace $a$ with $b$ and replace $b$ with $R$ and repeat steps $2$ and $3$ until $R=0$.
\end{enumerate} 
Examples:
\begin{enumerate}
\item Find the GCD of $126$ and $78$.
\begin{mdframed}[style=TheoremFrame]
\textit{Solution:}
\begin{align*}
126 &= 1\times 78 + 46\\
78 &= 1 \times 48 + 30\\
48 &= 1 \times 30 + 18\\
30 &= 1 \times 18 + 12\\
18 &= 1 \times 12 + 6\\
12 &= 2 \times 6 + 0
\end{align*}
Therefore GCD$(126,78)=6$.
\end{mdframed}
\item Find the GCD of $456$ and $112$.
\begin{mdframed}[style=TheoremFrame]
\textit{Solution:}
\begin{align*}
456 &= 4 \times 112 + 8\\
112 &= 14 \times 8 + 0
\end{align*}
Therefore GCD$(456,112)=8$.
\end{mdframed}
\newpage
\item Find the GCD of $97$ and $18$.
\begin{mdframed}[style=TheoremFrame]
\textit{Solution:}
\begin{align*}
97&=5 \times 18+7\\
18 &= 2 \times 7 + 4\\
7 &= 1 \times 4 + 3\\
4 &= 1 \times 3 + 1\\
3 &= 3 \times 1 + 0
\end{align*}
Therefore GCD$(97,18)=1$.
\end{mdframed}
\item Find the GCD of $234$ and $63$.
\begin{mdframed}[style=TheoremFrame]
\textit{Solution:}

\begin{align*}
234 &= 3 \times 63 + 45\\
63 &= 1 \times 45 + 18\\
45 &= 2 \times 18 + 9\\
18 &= 2 \times 9 + 0
\end{align*}
Therefore GCD$(234,63)=9$.
\end{mdframed}
\item Find the GCD of $140$ and $49$.
\begin{mdframed}[style=TheoremFrame]
\textit{Solution:}
\begin{align*}
140 &= 2 \times 49 + 42\\
49 &= 1 \times 42+7\\
42 &= 6\times 7 + 0
\end{align*}
Therefore GCD$(140,49)=7$.
\end{mdframed}
\end{enumerate}
\newpage
Questions:
\begin{enumerate}
\item Find the GCD of $412$ and $56$.
\begin{mdframed}[style=TheoremFrame]
\textit{Solution:}
\begin{align*}
412 &= 7 \times 56 + 20\\
56 &= 2 \times 20 + 16\\
20 &= 1 \times 16 + 4\\
16 &= 4 \times 4 + 0
\end{align*}
Therefore GCD$(412,56)=4$.
\end{mdframed}
\item Find the GCD of $414$ and $230$.
\begin{mdframed}[style=TheoremFrame]
\textit{Solution:}
\begin{align*}
414 &= 1 \times 230 + 184\\
230 &= 1\times 184 + 46\\
184 &= 4 \times 46
\end{align*}
Therefore GCD$(414,230)=46$.
\end{mdframed}
\item Find the GCD of $321$ and $144$.
\begin{mdframed}[style=TheoremFrame]
\textit{Solution:}
\begin{align*}
321 &= 2 \times 144 + 33\\
144 &= 4 \times 33 + 12\\
33 &= 2 \times 12 + 9\\
12 &= 1 \times 9 + 3\\
9 &= 3 \times 3 + 0
\end{align*}
Therefore GCD$(321,144)=3$.
\end{mdframed}
\newpage
\item Find the GCD of $1026$ and $453$.
\begin{mdframed}[style=TheoremFrame]
\textit{Solution:}
\begin{align*}
1026 &= 2 \times 453 + 120\\
453 &= 3 \times 120 + 93\\
120 &= 1 \times 93 + 27\\
93 &= 3 \times 27 + 12\\
27 &= 2 \times 12 + 3\\
12 &= 4 \times 3 + 0
\end{align*}
Therefore GCD$(1026,453)=3$.
\end{mdframed}
\item Find the GCD of $554$ and $214$.
\begin{mdframed}[style=TheoremFrame]
\textit{Solution:}
\begin{align*}
554 &= 2 \times 214 + 126\\
214 &= 1 \times 126 + 88\\
126 &= 1 \times 88 + 38\\
88 &= 2 \times 38 + 12\\
38 &= 3 \times 12 + 2\\
12 &= 6 \times 2 + 0
\end{align*}
Therefore GCD$(554,214)=2$.
\end{mdframed}
\item Find the GCD of $351$ and $189$.
\begin{mdframed}[style=TheoremFrame]
\textit{Solution:}
\begin{align*}
351 &= 1 \times 189 + 162\\
189 &= 1\times 162 + 27\\
162 &= 6 \times 27 + 0
\end{align*}
Therefore GCD$(351,189)=27$.
\end{mdframed}
\newpage
\item Find the GCD of $213$ and $47$.
\begin{mdframed}[style=TheoremFrame]
\textit{Solution:}
\begin{align*}
213 &= 4 \times 47 + 25\\
47 &= 1 \times 25 + 22\\
25 &= 1 \times 22 + 3\\
22 &= 7 \times 3 + 1\\
3 &= 3 \times 1 + 0
\end{align*}
Therefore GCD$(213,47)=1$.
\end{mdframed}
\item Find the GCD of $1452$ and $670$.
\begin{mdframed}[style=TheoremFrame]
\textit{Solution:}
\begin{align*}
1452 &= 2 \times 670 + 112\\
670 &= 5 \times 112 + 110\\
112 &= 1 \times 110 + 2\\
110 &= 55 \times 2 + 0
\end{align*}
Therefore GCD$(1452,670)=2$.
\end{mdframed}
\end{enumerate}
\end{document}